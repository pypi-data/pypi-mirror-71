\section*{Publications (Google Scholar H-index = 112)}

\subsection*{2020}

Aron AR, Ivry RB, Jeffery KJ, Poldrack RA, Schmidt R, Summerfield C, Urai AE (2020). How Can Neuroscientists Respond to the Climate Emergency? \textit{Neuron, 106}, 17-20. \href{http://dx.doi.org/10.1016/j.neuron.2020.02.019}{DOI} \vspace{2mm}

Bielczyk NZ et al. (2020). Effective Self-Management for Early Career Researchers in the Natural and Life Sciences. \textit{Neuron, 106}, 212-217. \href{https://osf.io/w6emk/}{OSF} \href{http://dx.doi.org/10.1016/j.neuron.2020.03.015}{DOI} \vspace{2mm}

Colenbier N, Van de Steen F, Uddin LQ, Poldrack RA, Calhoun VD, Marinazzo D (2020). Disambiguating the role of blood flow and global signal with partial information decomposition. \textit{Neuroimage, 213}, 116699. \href{http://dx.doi.org/10.1016/j.neuroimage.2020.116699}{DOI} \vspace{2mm}

Dockès J, Poldrack RA, Primet R, Gözükan H, Yarkoni T, Suchanek F, Thirion B, Varoquaux G (2020). NeuroQuery, comprehensive meta-analysis of human brain mapping. \textit{Elife, 9}. \href{https://www.ncbi.nlm.nih.gov/pmc/articles/PMC7164961}{OA} \href{http://dx.doi.org/10.7554/elife.53385}{DOI} \vspace{2mm}

Lurie DJ et al. (2020). Questions and controversies in the study of time-varying functional connectivity in resting fMRI. \textit{Netw Neurosci, 4}, 30-69. \href{https://www.ncbi.nlm.nih.gov/pmc/articles/PMC7006871}{OA} \href{https://osf.io/fa6zr/}{OSF} \href{http://dx.doi.org/10.1162/netn\_a\_00116}{DOI} \vspace{2mm}

Mazza GL et al. (2020). Correlation Database of 60 Cross-Disciplinary Surveys and Cognitive Tasks Assessing Self-Regulation. \textit{J Pers Assess}, 1-8. \href{https://github.com/IanEisenberg/Self\_Regulation\_Ontology/tree/master/Data}{Data} \href{http://dx.doi.org/10.1080/00223891.2020.1732994}{DOI} \vspace{2mm}

Mollon J et al. (2020). Cognitive impairment from early to middle adulthood in patients with affective and nonaffective psychotic disorders. \textit{Psychol Med, 50}, 48-57. \href{https://www.ncbi.nlm.nih.gov/pmc/articles/PMC7086288}{OA} \href{http://dx.doi.org/10.1017/s0033291718003938}{DOI} \vspace{2mm}

Thompson WH, Kastrati G, Finc K, Wright J, Shine JM, Poldrack RA (2020). Time-varying nodal measures with temporal community structure: A cautionary note to avoid misinterpretation. \textit{Hum Brain Mapp, 41}, 2347-2356. \href{http://dx.doi.org/10.1002/hbm.24950}{DOI} \vspace{2mm}

Thompson WH, Wright J, Bissett PG, Poldrack RA (2020). Dataset decay and the problem of sequential analyses on open datasets. \textit{Elife, 9}. \href{https://www.ncbi.nlm.nih.gov/pmc/articles/PMC7237204}{OA} \href{http://dx.doi.org/10.7554/elife.53498}{DOI} \vspace{2mm}

Tozzi L et al. (2020). The human connectome project for disordered emotional states: Protocol and rationale for a research domain criteria study of brain connectivity in young adult anxiety and depression. \textit{Neuroimage, 214}, 116715. \href{http://dx.doi.org/10.1016/j.neuroimage.2020.116715}{DOI} \vspace{2mm}

\subsection*{2019}

Aridan N, Malecek NJ, Poldrack RA, Schonberg T (2019). Neural correlates of effort-based valuation with prospective choices. \textit{Neuroimage, 185}, 446-454. \href{https://www.ncbi.nlm.nih.gov/pmc/articles/PMC6289638}{OA} \href{https://openneuro.org/datasets/ds001167/versions/00002}{Data} \href{http://dx.doi.org/10.1016/j.neuroimage.2018.10.051}{DOI} \vspace{2mm}

Botvinik-Nezer R, Iwanir R, Holzmeister F, Huber J, Johannesson M, Kirchler M, Dreber A, Camerer CF, Poldrack RA, Schonberg T (2019). fMRI data of mixed gambles from the Neuroimaging Analysis Replication and Prediction Study. \textit{Sci Data, 6}, 106. \href{https://www.ncbi.nlm.nih.gov/pmc/articles/PMC6602933}{OA} \href{https://openneuro.org/datasets/ds001734/versions/1.0.5}{Data} \href{http://dx.doi.org/10.1038/s41597-019-0113-7}{DOI} \vspace{2mm}

Davies G et al. (2019). Author Correction: Study of 300,486 individuals identifies 148 independent genetic loci influencing general cognitive function. \textit{Nat Commun, 10}, 2068. \href{https://www.ncbi.nlm.nih.gov/pmc/articles/PMC6494826}{OA} \href{http://dx.doi.org/10.1038/s41467-019-10160-w}{DOI} \vspace{2mm}

Eisenberg IW, Bissett PG, Zeynep Enkavi A, Li J, MacKinnon DP, Marsch LA, Poldrack RA (2019). Uncovering the structure of self-regulation through data-driven ontology discovery. \textit{Nat Commun, 10}, 2319. \href{https://www.ncbi.nlm.nih.gov/pmc/articles/PMC6534563}{OA} \href{https://github.com/IanEisenberg/Self\_Regulation\_Ontology}{Code} \href{https://github.com/IanEisenberg/Self\_Regulation\_Ontology/tree/master/Data}{Data} \href{https://osf.io/zk6w9/}{OSF} \href{http://dx.doi.org/10.1038/s41467-019-10301-1}{DOI} \vspace{2mm}

Enkavi AZ, Eisenberg IW, Bissett PG, Mazza GL, MacKinnon DP, Marsch LA, Poldrack RA (2019). Reply to Friedman and Banich: Right measures for the research question. \textit{Proc. Natl. Acad. Sci. U.S.A., 116}, 24398-24399. \href{https://www.ncbi.nlm.nih.gov/pmc/articles/PMC6900533}{OA} \href{http://dx.doi.org/10.1073/pnas.1917123116}{DOI} \vspace{2mm}

Enkavi AZ, Eisenberg IW, Bissett PG, Mazza GL, MacKinnon DP, Marsch LA, Poldrack RA (2019). Large-scale analysis of test-retest reliabilities of self-regulation measures. \textit{Proc. Natl. Acad. Sci. U.S.A., 116}, 5472-5477. \href{https://www.ncbi.nlm.nih.gov/pmc/articles/PMC6431228}{OA} \href{https://github.com/IanEisenberg/Self\_Regulation\_Ontology/tree/master/Data}{Data} \href{https://osf.io/5mjns/}{OSF} \href{http://dx.doi.org/10.1073/pnas.1818430116}{DOI} \vspace{2mm}

Esteban O et al. (2019). fMRIPrep: a robust preprocessing pipeline for functional MRI. \textit{Nat. Methods, 16}, 111-116. \href{https://www.ncbi.nlm.nih.gov/pmc/articles/PMC6319393}{OA} \href{http://dx.doi.org/10.1038/s41592-018-0235-4}{DOI} \vspace{2mm}

Esteban O, Blair RW, Nielson DM, Varada JC, Marrett S, Thomas AG, Poldrack RA, Gorgolewski KJ (2019). Crowdsourced MRI quality metrics and expert quality annotations for training of humans and machines. \textit{Sci Data, 6}, 30. \href{https://www.ncbi.nlm.nih.gov/pmc/articles/PMC6472378}{OA} \href{http://dx.doi.org/10.1038/s41597-019-0035-4}{DOI} \vspace{2mm}

Kebets V, Holmes AJ, Orban C, Tang S, Li J, Sun N, Kong R, Poldrack RA, Yeo BTT (2019). Somatosensory-Motor Dysconnectivity Spans Multiple Transdiagnostic Dimensions of Psychopathology. \textit{Biol. Psychiatry, 86}, 779-791. \href{http://dx.doi.org/10.1016/j.biopsych.2019.06.013}{DOI} \vspace{2mm}

King M, Hernandez-Castillo CR, Poldrack RA, Ivry RB, Diedrichsen J (2019). Functional boundaries in the human cerebellum revealed by a multi-domain task battery. \textit{Nat. Neurosci., 22}, 1371-1378. \href{https://openneuro.org/datasets/ds002105/versions/1.1.0}{Data} \href{http://dx.doi.org/10.1038/s41593-019-0436-x}{DOI} \vspace{2mm}

Lam M et al. (2019). Pleiotropic Meta-Analysis of Cognition, Education, and Schizophrenia Differentiates Roles of Early Neurodevelopmental and Adult Synaptic Pathways. \textit{Am. J. Hum. Genet., 105}, 334-350. \href{https://www.ncbi.nlm.nih.gov/pmc/articles/PMC6699140}{OA} \href{http://dx.doi.org/10.1016/j.ajhg.2019.06.012}{DOI} \vspace{2mm}

Li M, Han Y, Aburn MJ, Breakspear M, Poldrack RA, Shine JM, Lizier JT (2019). Transitions in information processing dynamics at the whole-brain network level are driven by alterations in neural gain. \textit{PLoS Comput. Biol., 15}, e1006957. \href{https://www.ncbi.nlm.nih.gov/pmc/articles/PMC6793849}{OA} \href{http://dx.doi.org/10.1371/journal.pcbi.1006957}{DOI} \vspace{2mm}

Manapat PD, Edwards MC, MacKinnon DP, Poldrack RA, Marsch LA (2019). A Psychometric Analysis of the Brief Self-Control Scale. \textit{Assessment}, 1073191119890021. \href{http://dx.doi.org/10.1177/1073191119890021}{DOI} \vspace{2mm}

Poldrack RA (2019). The Costs of Reproducibility. \textit{Neuron, 101}, 11-14. \href{http://dx.doi.org/10.1016/j.neuron.2018.11.030}{DOI} \vspace{2mm}

Poldrack RA et al. (2019). The importance of standards for sharing of computational models and data. \textit{Comput Brain Behav, 2}, 229-232. \href{https://www.ncbi.nlm.nih.gov/pmc/articles/PMC7241435}{OA} \href{http://dx.doi.org/10.1007/s42113-019-00062-x}{DOI} \vspace{2mm}

Poldrack RA, Huckins G, Varoquaux G (2019). Establishment of Best Practices for Evidence for Prediction: A Review. \textit{JAMA Psychiatry, 77}, 534. \href{http://dx.doi.org/10.1001/jamapsychiatry.2019.3671}{DOI} \vspace{2mm}

Poldrack RA, Whitaker K, Kennedy D (2019). Introduction to the special issue on reproducibility in neuroimaging. \textit{Neuroimage}, 116357. \href{http://dx.doi.org/10.1016/j.neuroimage.2019.116357}{DOI} \vspace{2mm}

Reid AT et al. (2019). Advancing functional connectivity research from association to causation. \textit{Nat. Neurosci., 22}, 1751-1760. \href{http://dx.doi.org/10.1038/s41593-019-0510-4}{DOI} \vspace{2mm}

Shine JM, Bell PT, Matar E, Poldrack RA, Lewis SJG, Halliday GM, O'Callaghan C (2019). Dopamine depletion alters macroscopic network dynamics in Parkinson's disease. \textit{Brain, 142}, 1024-1034. \href{https://www.ncbi.nlm.nih.gov/pmc/articles/PMC6904322}{OA} \href{http://dx.doi.org/10.1093/brain/awz034}{DOI} \vspace{2mm}

Shine JM, Breakspear M, Bell PT, Ehgoetz Martens KA, Shine R, Koyejo O, Sporns O, Poldrack RA (2019). Publisher Correction: Human cognition involves the dynamic integration of neural activity and neuromodulatory systems. \textit{Nat. Neurosci., 22}, 1036. \href{http://dx.doi.org/10.1038/s41593-019-0347-x}{DOI} \vspace{2mm}

Shine JM, Breakspear M, Bell PT, Ehgoetz Martens KA, Shine R, Koyejo O, Sporns O, Poldrack RA (2019). Human cognition involves the dynamic integration of neural activity and neuromodulatory systems. \textit{Nat. Neurosci., 22}, 289-296. \href{http://dx.doi.org/10.1038/s41593-018-0312-0}{DOI} \vspace{2mm}

Shine JM, Hearne LJ, Breakspear M, Hwang K, Müller EJ, Sporns O, Poldrack RA, Mattingley JB, Cocchi L (2019). The Low-Dimensional Neural Architecture of Cognitive Complexity Is Related to Activity in Medial Thalamic Nuclei. \textit{Neuron, 104}, 849-855.e3. \href{http://dx.doi.org/10.1016/j.neuron.2019.09.002}{DOI} \vspace{2mm}

Smeets PAM, Dagher A, Hare TA, Kullmann S, van der Laan LN, Poldrack RA, Preissl H, Small D, Stice E, Veldhuizen MG (2019). Good practice in food-related neuroimaging. \textit{Am. J. Clin. Nutr., 109}, 491-503. \href{http://dx.doi.org/10.1093/ajcn/nqy344}{DOI} \vspace{2mm}

Varoquaux G, Poldrack RA (2019). Predictive models avoid excessive reductionism in cognitive neuroimaging. \textit{Curr. Opin. Neurobiol., 55}, 1-6. \href{http://dx.doi.org/10.1016/j.conb.2018.11.002}{DOI} \vspace{2mm}

Verbruggen F et al. (2019). A consensus guide to capturing the ability to inhibit actions and impulsive behaviors in the stop-signal task. \textit{Elife, 8}. \href{https://www.ncbi.nlm.nih.gov/pmc/articles/PMC6533084}{OA} \href{http://dx.doi.org/10.7554/elife.46323}{DOI} \vspace{2mm}

Zuo XN, Biswal BB, Poldrack RA (2019). Editorial: Reliability and Reproducibility in Functional Connectomics. \textit{Front Neurosci, 13}, 117. \href{https://www.ncbi.nlm.nih.gov/pmc/articles/PMC6391345}{OA} \href{http://dx.doi.org/10.3389/fnins.2019.00117}{DOI} \vspace{2mm}

Zuo XN, Biswal BB, Poldrack RA (2019). Corrigendum: Editorial: Reliability and Reproducibility in Functional Connectomics. \textit{Front Neurosci, 13}, 374. \href{https://www.ncbi.nlm.nih.gov/pmc/articles/PMC6477511}{OA} \href{http://dx.doi.org/10.3389/fnins.2019.00374}{DOI} \vspace{2mm}

\subsection*{2018}

Bakkour A, Botvinik-Nezer R, Cohen N, Hover AM, Poldrack RA, Schonberg T (2018). Spacing of cue-approach training leads to better maintenance of behavioral change. \textit{PLoS ONE, 13}, e0201580. \href{https://www.ncbi.nlm.nih.gov/pmc/articles/PMC6066248}{OA} \href{https://osf.io/fdvrk/}{OSF} \href{http://dx.doi.org/10.1371/journal.pone.0201580}{DOI} \vspace{2mm}

Davies G et al. (2018). Study of 300,486 individuals identifies 148 independent genetic loci influencing general cognitive function. \textit{Nat Commun, 9}, 2098. \href{https://www.ncbi.nlm.nih.gov/pmc/articles/PMC5974083}{OA} \href{http://dx.doi.org/10.1038/s41467-018-04362-x}{DOI} \vspace{2mm}

Dockès J, Wassermann D, Poldrack R, Suchanek F, Thirion B, Varoquaux G (2018). Text to Brain: Predicting the Spatial Distribution of Neuroimaging Observations from Text Reports. In \textit{Medical Image Computing and Computer Assisted Intervention – MICCAI 2018.} (p. 584-592). Springer International Publishing. \href{http://dx.doi.org/10.1007/978-3-030-00931-1\_67}{DOI} \vspace{2mm}

Eisenberg IW et al. (2018). Applying novel technologies and methods to inform the ontology of self-regulation. \textit{Behav Res Ther, 101}, 46-57. \href{https://www.ncbi.nlm.nih.gov/pmc/articles/PMC5801197}{OA} \href{https://osf.io/amxpv/}{OSF} \href{http://dx.doi.org/10.1016/j.brat.2017.09.014}{DOI} \vspace{2mm}

Esteban O, Poldrack RA, Gorgolewski KJ (2018). Improving Out-of-Sample Prediction of Quality of MRIQC. In \textit{Intravascular Imaging and Computer Assisted Stenting and Large-Scale Annotation of Biomedical Data and Expert Label Synthesis.} (p. 190-199). Springer International Publishing. \href{http://dx.doi.org/10.1007/978-3-030-01364-6\_21}{DOI} \vspace{2mm}

Gorgolewski KJ, Nichols T, Kennedy DN, Poline JB, Poldrack RA (2018). Making replication prestigious. \textit{Behav Brain Sci, 41}, e131. \href{http://dx.doi.org/10.1017/s0140525x18000663}{DOI} \vspace{2mm}

Lam M et al. (2018). Multi-Trait Analysis of GWAS and Biological Insights Into Cognition: A Response to Hill (2018). \textit{Twin Res Hum Genet, 21}, 394-397. \href{https://osf.io/t5kzx/}{OSF} \href{http://dx.doi.org/10.1017/thg.2018.46}{DOI} \vspace{2mm}

Mathias SR, Knowles EEM, Barrett J, Beetham T, Leach O, Buccheri S, Aberizk K, Blangero J, Poldrack RA, Glahn DC (2018). Deficits in visual working-memory capacity and general cognition in African Americans with psychosis. \textit{Schizophr. Res., 193}, 100-106. \href{https://www.ncbi.nlm.nih.gov/pmc/articles/PMC5825248}{OA} \href{http://dx.doi.org/10.1016/j.schres.2017.08.015}{DOI} \vspace{2mm}

Naselaris T, Bassett DS, Fletcher AK, Kording K, Kriegeskorte N, Nienborg H, Poldrack RA, Shohamy D, Kay K (2018). Cognitive Computational Neuroscience: A New Conference for an Emerging Discipline. \textit{Trends Cogn. Sci. (Regul. Ed.), 22}, 365-367. \href{https://www.ncbi.nlm.nih.gov/pmc/articles/PMC5911192}{OA} \href{http://dx.doi.org/10.1016/j.tics.2018.02.008}{DOI} \vspace{2mm}

Poldrack RA (2018).  \textit{The New Mind Readers: What Neuroimaging Can and Cannot Reveal about our Thoughts}. Princeton University Press. \vspace{2mm}

Poldrack RA (2018).  \textit{Statistical Thinking for the 21st Century}. http://statsthinking21.org. \vspace{2mm}

Poldrack RA, Monahan J, Imrey PB, Reyna V, Raichle ME, Faigman D, Buckholtz JW (2018). Predicting Violent Behavior: What Can Neuroscience Add? \textit{Trends Cogn. Sci. (Regul. Ed.), 22}, 111-123. \href{https://www.ncbi.nlm.nih.gov/pmc/articles/PMC5794654}{OA} \href{https://osf.io/tgknp/}{OSF} \href{http://dx.doi.org/10.1016/j.tics.2017.11.003}{DOI} \vspace{2mm}

Savage JE et al. (2018). Genome-wide association meta-analysis in 269,867 individuals identifies new genetic and functional links to intelligence. \textit{Nat. Genet., 50}, 912-919. \href{https://www.ncbi.nlm.nih.gov/pmc/articles/PMC6411041}{OA} \href{http://dx.doi.org/10.1038/s41588-018-0152-6}{DOI} \vspace{2mm}

Shine JM, Aburn MJ, Breakspear M, Poldrack RA (2018). The modulation of neural gain facilitates a transition between functional segregation and integration in the brain. \textit{Elife, 7}. \href{https://www.ncbi.nlm.nih.gov/pmc/articles/PMC5818252}{OA} \href{http://dx.doi.org/10.7554/elife.31130}{DOI} \vspace{2mm}

Shine JM, Poldrack RA (2018). Principles of dynamic network reconfiguration across diverse brain states. \textit{Neuroimage, 180}, 396-405. \href{http://dx.doi.org/10.1016/j.neuroimage.2017.08.010}{DOI} \vspace{2mm}

Shine JM, van den Brink RL, Hernaus D, Nieuwenhuis S, Poldrack RA (2018). Catecholaminergic manipulation alters dynamic network topology across cognitive states. \textit{Netw Neurosci, 2}, 381-396. \href{https://www.ncbi.nlm.nih.gov/pmc/articles/PMC6145851}{OA} \href{http://dx.doi.org/10.1162/netn\_a\_00042}{DOI} \vspace{2mm}

Tansey W, Koyejo O, Poldrack RA, Scott JG (2018). False Discovery Rate Smoothing \textit{Journal of the American Statistical Association, 113}, 1156-1171. \href{http://dx.doi.org/10.1080/01621459.2017.1319838}{DOI} \vspace{2mm}

Varoquaux G, Schwartz Y, Poldrack RA, Gauthier B, Bzdok D, Poline JB, Thirion B (2018). Atlases of cognition with large-scale human brain mapping. \textit{PLoS Comput. Biol., 14}, e1006565. \href{https://www.ncbi.nlm.nih.gov/pmc/articles/PMC6289578}{OA} \href{http://dx.doi.org/10.1371/journal.pcbi.1006565}{DOI} \vspace{2mm}

White C, Poldrack RA (2018). Methods for fMRI analysis.. In \textit{The Stevens’ Handbook of Experimental Psychology and Cognitive Neuroscience, Fourth Edition (Volume 5).} Wiley. \href{http://dx.doi.org/10.1002/9781119170174.epcn515}{DOI} \vspace{2mm}

Wimmer GE, Li JK, Gorgolewski KJ, Poldrack RA (2018). Reward Learning over Weeks Versus Minutes Increases the Neural Representation of Value in the Human Brain. \textit{J. Neurosci., 38}, 7649-7666. \href{https://www.ncbi.nlm.nih.gov/pmc/articles/PMC6113901}{OA} \href{https://osf.io/z2gwf/}{OSF} \href{http://dx.doi.org/10.1523/jneurosci.0075-18.2018}{DOI} \vspace{2mm}

\subsection*{2017}

Acikalin MY, Gorgolewski KJ, Poldrack RA (2017). A Coordinate-Based Meta-Analysis of Overlaps in Regional Specialization and Functional Connectivity across Subjective Value and Default Mode Networks. \textit{Front Neurosci, 11}, 1. \href{https://www.ncbi.nlm.nih.gov/pmc/articles/PMC5243799}{OA} \href{http://dx.doi.org/10.3389/fnins.2017.00001}{DOI} \vspace{2mm}

Bakkour A, Lewis-Peacock JA, Poldrack RA, Schonberg T (2017). Neural mechanisms of cue-approach training. \textit{Neuroimage, 151}, 92-104. \href{https://www.ncbi.nlm.nih.gov/pmc/articles/PMC5365383}{OA} \href{http://dx.doi.org/10.1016/j.neuroimage.2016.09.059}{DOI} \vspace{2mm}

Eckert MA, Vaden KI, Maxwell AB, Cute SL, Gebregziabher M, Berninger VW (2017). Common Brain Structure Findings Across Children with Varied Reading Disability Profiles. \textit{Sci Rep, 7}, 6009. \href{https://www.ncbi.nlm.nih.gov/pmc/articles/PMC5519686}{OA} \href{http://dx.doi.org/10.1038/s41598-017-05691-5}{DOI} \vspace{2mm}

Esteban O, Birman D, Schaer M, Koyejo OO, Poldrack RA, Gorgolewski KJ (2017). MRIQC: Advancing the automatic prediction of image quality in MRI from unseen sites. \textit{PLoS ONE, 12}, e0184661. \href{https://www.ncbi.nlm.nih.gov/pmc/articles/PMC5612458}{OA} \href{https://osf.io/haf97/}{OSF} \href{http://dx.doi.org/10.1371/journal.pone.0184661}{DOI} \vspace{2mm}

Gilron R, Rosenblatt J, Koyejo O, Poldrack RA, Mukamel R (2017). What's in a pattern? Examining the type of signal multivariate analysis uncovers at the group level. \textit{Neuroimage, 146}, 113-120. \href{http://dx.doi.org/10.1016/j.neuroimage.2016.11.019}{DOI} \vspace{2mm}

Gorgolewski KJ et al. (2017). BIDS apps: Improving ease of use, accessibility, and reproducibility of neuroimaging data analysis methods. \textit{PLoS Comput. Biol., 13}, e1005209. \href{https://www.ncbi.nlm.nih.gov/pmc/articles/PMC5363996}{OA} \href{http://dx.doi.org/10.1371/journal.pcbi.1005209}{DOI} \vspace{2mm}

Gorgolewski KJ, Durnez J, Poldrack RA (2017). Preprocessed Consortium for Neuropsychiatric Phenomics dataset. \textit{F1000Res, 6}, 1262. \href{https://www.ncbi.nlm.nih.gov/pmc/articles/PMC5664981}{OA} \href{http://dx.doi.org/10.12688/f1000research.11964.2}{DOI} \vspace{2mm}

Hodgson K et al. (2017). Shared Genetic Factors Influence Head Motion During MRI and Body Mass Index. \textit{Cereb. Cortex, 27}, 5539-5546. \href{https://www.ncbi.nlm.nih.gov/pmc/articles/PMC6075600}{OA} \href{http://dx.doi.org/10.1093/cercor/bhw321}{DOI} \vspace{2mm}

Khanna R, Ghosh J, Poldrack RA,  Koyejo O (2017). A Deflation Method for Structured Probabilistic PCA. \textit{Proceedings of the 2017 SIAM International Conference on Data Mining, }. \href{http://dx.doi.org/10.1137/1.9781611974973.60}{DOI} \vspace{2mm}

Kiar G et al. (2017). Science in the cloud (SIC): A use case in MRI connectomics. \textit{Gigascience, 6}, 1-10. \href{https://www.ncbi.nlm.nih.gov/pmc/articles/PMC5467033}{OA} \href{http://dx.doi.org/10.1093/gigascience/gix013}{DOI} \vspace{2mm}

Lam M et al. (2017). Large-Scale Cognitive GWAS Meta-Analysis Reveals Tissue-Specific Neural Expression and Potential Nootropic Drug Targets. \textit{Cell Rep, 21}, 2597-2613. \href{https://www.ncbi.nlm.nih.gov/pmc/articles/PMC5789458}{OA} \href{http://dx.doi.org/10.1016/j.celrep.2017.11.028}{DOI} \vspace{2mm}

Lenartowicz A, Poldrack R (2017). Brain Imaging☆. In \textit{Reference Module in Neuroscience and Biobehavioral Psychology.} Elsevier. \href{http://dx.doi.org/10.1016/b978-0-12-809324-5.00274-1}{DOI} \vspace{2mm}

Mathias SR, Knowles EEM, Barrett J, Leach O, Buccheri S, Beetham T, Blangero J, Poldrack RA, Glahn DC (2017). The Processing-Speed Impairment in Psychosis Is More Than Just Accelerated Aging. \textit{Schizophr Bull, 43}, 814-823. \href{https://www.ncbi.nlm.nih.gov/pmc/articles/PMC5472152}{OA} \href{http://dx.doi.org/10.1093/schbul/sbw168}{DOI} \vspace{2mm}

Nichols TE et al. (2017). Best practices in data analysis and sharing in neuroimaging using MRI. \textit{Nat. Neurosci., 20}, 299-303. \href{https://www.ncbi.nlm.nih.gov/pmc/articles/PMC5685169}{OA} \href{http://dx.doi.org/10.1038/nn.4500}{DOI} \vspace{2mm}

Poldrack R (2017). Neuroscience: The risks of reading the brain \textit{Nature, 541}, 156-156. \href{http://dx.doi.org/10.1038/541156a}{DOI} \vspace{2mm}

Poldrack RA (2017). Developing a reproducible workfow for large-scale phenotyping.. In \textit{The Practice of Reproducible Research: Case Studies and Lessons from the Data-Intensive Sciences.} Oakland, CA: University of California Press. \href{http://dx.doi.org/9780520294752}{DOI} \vspace{2mm}

Poldrack RA (2017). Precision Neuroscience: Dense Sampling of Individual Brains. \textit{Neuron, 95}, 727-729. \href{http://dx.doi.org/10.1016/j.neuron.2017.08.002}{DOI} \vspace{2mm}

Poldrack RA, Baker CI, Durnez J, Gorgolewski KJ, Matthews PM, Munafò MR, Nichols TE, Poline JB, Vul E, Yarkoni T (2017). Scanning the horizon: towards transparent and reproducible neuroimaging research. \textit{Nat. Rev. Neurosci., 18}, 115-126. \href{https://www.ncbi.nlm.nih.gov/pmc/articles/PMC6910649}{OA} \href{https://osf.io/spr9a/}{OSF} \href{http://dx.doi.org/10.1038/nrn.2016.167}{DOI} \vspace{2mm}

Poldrack RA, Gorgolewski KJ (2017). OpenfMRI: Open sharing of task fMRI data. \textit{Neuroimage, 144}, 259-261. \href{https://www.ncbi.nlm.nih.gov/pmc/articles/PMC4669234}{OA} \href{http://dx.doi.org/10.1016/j.neuroimage.2015.05.073}{DOI} \vspace{2mm}

Rubin TN, Koyejo O, Gorgolewski KJ, Jones MN, Poldrack RA, Yarkoni T (2017). Decoding brain activity using a large-scale probabilistic functional-anatomical atlas of human cognition. \textit{PLoS Comput. Biol., 13}, e1005649. \href{https://www.ncbi.nlm.nih.gov/pmc/articles/PMC5683652}{OA} \href{http://dx.doi.org/10.1371/journal.pcbi.1005649}{DOI} \vspace{2mm}

Shine JM, Kucyi A, Foster BL, Bickel S, Wang D, Liu H, Poldrack RA, Hsieh LT, Hsiang JC, Parvizi J (2017). Distinct Patterns of Temporal and Directional Connectivity among Intrinsic Networks in the Human Brain. \textit{J. Neurosci., 37}, 9667-9674. \href{https://www.ncbi.nlm.nih.gov/pmc/articles/PMC6596608}{OA} \href{http://dx.doi.org/10.1523/jneurosci.1574-17.2017}{DOI} \vspace{2mm}

Trampush JW et al. (2017). GWAS meta-analysis reveals novel loci and genetic correlates for general cognitive function: a report from the COGENT consortium. \textit{Mol. Psychiatry, 22}, 336-345. \href{https://www.ncbi.nlm.nih.gov/pmc/articles/PMC5322272}{OA} \href{http://dx.doi.org/10.1038/mp.2016.244}{DOI} \vspace{2mm}

Trampush JW et al. (2017). GWAS meta-analysis reveals novel loci and genetic correlates for general cognitive function: a report from the COGENT consortium. \textit{Mol. Psychiatry, 22}, 1651-1652. \href{https://www.ncbi.nlm.nih.gov/pmc/articles/PMC5659072}{OA} \href{http://dx.doi.org/10.1038/mp.2017.197}{DOI} \vspace{2mm}

Xiao X, Dong Q, Gao J, Men W, Poldrack RA, Xue G (2017). Transformed Neural Pattern Reinstatement during Episodic Memory Retrieval. \textit{J. Neurosci., 37}, 2986-2998. \href{https://www.ncbi.nlm.nih.gov/pmc/articles/PMC6596730}{OA} \href{http://dx.doi.org/10.1523/jneurosci.2324-16.2017}{DOI} \vspace{2mm}

\subsection*{2016}

Bakkour A, Leuker C, Hover AM, Giles N, Poldrack RA, Schonberg T (2016). Mechanisms of Choice Behavior Shift Using Cue-approach Training. \textit{Front Psychol, 7}, 421. \href{https://www.ncbi.nlm.nih.gov/pmc/articles/PMC4804288}{OA} \href{http://dx.doi.org/10.3389/fpsyg.2016.00421}{DOI} \vspace{2mm}

Eckert MA, Berninger VW, Hoeft F, Vaden KI (2016). A case of Bilateral Perisylvian Syndrome with reading disability \textit{Cortex, 76}, 121-124. \href{http://dx.doi.org/10.1016/j.cortex.2016.01.004}{DOI} \vspace{2mm}

Eisenberg I, Poldrack RA (2016). Task-set Selection in Probabilistic Environments: a Model of Task-set Inference. \textit{Proceedings of Cognitive Science Society, }. \href{http://dx.doi.org/lorafmrm}{DOI} \vspace{2mm}

Gorgolewski KJ et al. (2016). NeuroVault.org: A repository for sharing unthresholded statistical maps, parcellations, and atlases of the human brain. \textit{Neuroimage, 124}, 1242-1244. \href{https://www.ncbi.nlm.nih.gov/pmc/articles/PMC4806527}{OA} \href{http://dx.doi.org/10.1016/j.neuroimage.2015.04.016}{DOI} \vspace{2mm}

Gorgolewski KJ et al. (2016). The brain imaging data structure, a format for organizing and describing outputs of neuroimaging experiments. \textit{Sci Data, 3}, 160044. \href{https://www.ncbi.nlm.nih.gov/pmc/articles/PMC4978148}{OA} \href{http://dx.doi.org/10.1038/sdata.2016.44}{DOI} \vspace{2mm}

Gorgolewski KJ, Poldrack RA (2016). A Practical Guide for Improving Transparency and Reproducibility in Neuroimaging Research. \textit{PLoS Biol., 14}, e1002506. \href{https://www.ncbi.nlm.nih.gov/pmc/articles/PMC4936733}{OA} \href{http://dx.doi.org/10.1371/journal.pbio.1002506}{DOI} \vspace{2mm}

Hastings J et al. (2016). Interdyscyplinarne perspektywy rozwoju, integracji i zastosowań ontologii poznawczych \textit{AVANT. The Journal of the Philosophical-Interdisciplinary Vanguard, VII}, 101-117. \href{http://dx.doi.org/10.26913/70302016.0109.0007}{DOI} \vspace{2mm}

Patterson TK, Lenartowicz A, Berkman ET, Ji D, Poldrack RA, Knowlton BJ (2016). Putting the brakes on the brakes: negative emotion disrupts cognitive control network functioning and alters subsequent stopping ability. \textit{Exp Brain Res, 234}, 3107-3118. \href{https://www.ncbi.nlm.nih.gov/pmc/articles/PMC5073018}{OA} \href{http://dx.doi.org/10.1007/s00221-016-4709-2}{DOI} \vspace{2mm}

Poldrack RA et al. (2016). A phenome-wide examination of neural and cognitive function. \textit{Sci Data, 3}, 160110. \href{https://www.ncbi.nlm.nih.gov/pmc/articles/PMC5139672}{OA} \href{https://openneuro.org/datasets/ds000030/versions/1.0.0}{Data} \href{http://dx.doi.org/10.1038/sdata.2016.110}{DOI} \vspace{2mm}

Poldrack RA, Yarkoni T (2016). From Brain Maps to Cognitive Ontologies: Informatics and the Search for Mental Structure. \textit{Annu Rev Psychol, 67}, 587-612. \href{https://www.ncbi.nlm.nih.gov/pmc/articles/PMC4701616}{OA} \href{http://dx.doi.org/10.1146/annurev-psych-122414-033729}{DOI} \vspace{2mm}

Shine JM, Bissett PG, Bell PT, Koyejo O, Balsters JH, Gorgolewski KJ, Moodie CA, Poldrack RA (2016). The Dynamics of Functional Brain Networks: Integrated Network States during Cognitive Task Performance. \textit{Neuron, 92}, 544-554. \href{https://www.ncbi.nlm.nih.gov/pmc/articles/PMC5073034}{OA} \href{http://dx.doi.org/10.1016/j.neuron.2016.09.018}{DOI} \vspace{2mm}

Shine JM, Eisenberg I, Poldrack RA (2016). Computational specificity in the human brain. \textit{Behav Brain Sci, 39}, e131. \href{http://dx.doi.org/10.1017/s0140525x1500165x}{DOI} \vspace{2mm}

Shine JM, Koyejo O, Poldrack RA (2016). Temporal metastates are associated with differential patterns of time-resolved connectivity, network topology, and attention. \textit{Proc. Natl. Acad. Sci. U.S.A., 113}, 9888-91. \href{https://www.ncbi.nlm.nih.gov/pmc/articles/PMC5024627}{OA} \href{http://dx.doi.org/10.1073/pnas.1604898113}{DOI} \vspace{2mm}

Sochat VV, Eisenberg IW, Enkavi AZ, Li J, Bissett PG, Poldrack RA (2016). The Experiment Factory: Standardizing Behavioral Experiments. \textit{Front Psychol, 7}, 610. \href{https://www.ncbi.nlm.nih.gov/pmc/articles/PMC4844768}{OA} \href{http://dx.doi.org/10.3389/fpsyg.2016.00610}{DOI} \vspace{2mm}

Wager TD, Atlas LY, Botvinick MM, Chang LJ, Coghill RC, Davis KD, Iannetti GD, Poldrack RA, Shackman AJ, Yarkoni T (2016). Pain in the ACC? \textit{Proc. Natl. Acad. Sci. U.S.A., 113}, E2474-5. \href{https://www.ncbi.nlm.nih.gov/pmc/articles/PMC4983860}{OA} \href{http://dx.doi.org/10.1073/pnas.1600282113}{DOI} \vspace{2mm}

Wiener M, Sommer FT, Ives ZG, Poldrack RA, Litt B (2016). Enabling an Open Data Ecosystem for the Neurosciences. \textit{Neuron, 92}, 617-621. \href{http://dx.doi.org/10.1016/j.neuron.2016.10.037}{DOI} \vspace{2mm}

Wiener M, Sommer FT, Ives ZG, Poldrack RA, Litt B (2016). Enabling an Open Data Ecosystem for the Neurosciences. \textit{Neuron, 92}, 929. \href{http://dx.doi.org/10.1016/j.neuron.2016.11.009}{DOI} \vspace{2mm}

Worthy DA, Davis T, Gorlick MA, Cooper JA, Bakkour A, Mumford JA, Poldrack RA, Todd Maddox W (2016). Neural correlates of state-based decision-making in younger and older adults. \textit{Neuroimage, 130}, 13-23. \href{https://www.ncbi.nlm.nih.gov/pmc/articles/PMC4808466}{OA} \href{http://dx.doi.org/10.1016/j.neuroimage.2015.12.004}{DOI} \vspace{2mm}

\subsection*{2015}

Chen MY, Jimura K, White CN, Maddox WT, Poldrack RA (2015). Multiple brain networks contribute to the acquisition of bias in perceptual decision-making. \textit{Front Neurosci, 9}, 63. \href{https://www.ncbi.nlm.nih.gov/pmc/articles/PMC4350407}{OA} \href{http://dx.doi.org/10.3389/fnins.2015.00063}{DOI} \vspace{2mm}

Gorgolewski KJ et al. (2015). NeuroVault.org: a web-based repository for collecting and sharing unthresholded statistical maps of the human brain. \textit{Front Neuroinform, 9}, 8. \href{https://www.ncbi.nlm.nih.gov/pmc/articles/PMC4392315}{OA} \href{http://dx.doi.org/10.3389/fninf.2015.00008}{DOI} \vspace{2mm}

Helfinstein SM, Mumford JA, Poldrack RA (2015). If all your friends jumped off a bridge: the effect of others' actions on engagement in and recommendation of risky behaviors. \textit{J Exp Psychol Gen, 144}, 12-7. \href{http://dx.doi.org/10.1037/xge0000043}{DOI} \vspace{2mm}

Khanna R, Ghosh J, Poldrack RA, Koyejo O (2015). Sparse submodular probabilistic PCA. \textit{Proceedings of the 18th International conference on Artificial Intelligence and Statistics (AISTATS), }. \href{http://dx.doi.org/pnawvqzs}{DOI} \vspace{2mm}

Laumann TO et al. (2015). Functional System and Areal Organization of a Highly Sampled Individual Human Brain. \textit{Neuron, 87}, 657-70. \href{https://www.ncbi.nlm.nih.gov/pmc/articles/PMC4642864}{OA} \href{https://openneuro.org/datasets/ds000031/versions/00001}{Data} \href{http://dx.doi.org/10.1016/j.neuron.2015.06.037}{DOI} \vspace{2mm}

Mumford JA, Poline JB, Poldrack RA (2015). Orthogonalization of regressors in FMRI models. \textit{PLoS ONE, 10}, e0126255. \href{https://www.ncbi.nlm.nih.gov/pmc/articles/PMC4412813}{OA} \href{http://dx.doi.org/10.1371/journal.pone.0126255}{DOI} \vspace{2mm}

Poldrack R (2015). Introduction to Cognitive Neuroscience. In \textit{Brain Mapping.} (p. 259-260). Elsevier. \href{http://dx.doi.org/10.1016/b978-0-12-397025-1.09990-5}{DOI} \vspace{2mm}

Poldrack R (2015). Reverse Inference. In \textit{Brain Mapping.} (p. 647-650). Elsevier. \href{http://dx.doi.org/10.1016/b978-0-12-397025-1.00346-8}{DOI} \vspace{2mm}

Poldrack RA (2015). Is "efficiency" a useful concept in cognitive neuroscience? \textit{Dev Cogn Neurosci, 11}, 12-7. \href{https://www.ncbi.nlm.nih.gov/pmc/articles/PMC6989750}{OA} \href{https://github.com/poldrack/rtmodel}{Code} \href{http://dx.doi.org/10.1016/j.dcn.2014.06.001}{DOI} \vspace{2mm}

Poldrack RA et al. (2015). Long-term neural and physiological phenotyping of a single human. \textit{Nat Commun, 6}, 8885. \href{https://www.ncbi.nlm.nih.gov/pmc/articles/PMC4682164}{OA} \href{https://github.com/poldrack/myconnectome}{Code} \href{https://openneuro.org/datasets/ds000031/versions/00001}{Data} \href{http://dx.doi.org/10.1038/ncomms9885}{DOI} \vspace{2mm}

Poldrack RA, Farah MJ (2015). Progress and challenges in probing the human brain. \textit{Nature, 526}, 371-9. \href{http://dx.doi.org/10.1038/nature15692}{DOI} \vspace{2mm}

Poldrack RA, Poline JB (2015). The publication and reproducibility challenges of shared data. \textit{Trends Cogn. Sci. (Regul. Ed.), 19}, 59-61. \href{http://dx.doi.org/10.1016/j.tics.2014.11.008}{DOI} \vspace{2mm}

Shine JM, Koyejo O, Bell PT, Gorgolewski KJ, Gilat M, Poldrack RA (2015). Estimation of dynamic functional connectivity using Multiplication of Temporal Derivatives. \textit{Neuroimage, 122}, 399-407. \href{http://dx.doi.org/10.1016/j.neuroimage.2015.07.064}{DOI} \vspace{2mm}

Sochat VV, Gorgolewski KJ, Koyejo O, Durnez J, Poldrack RA (2015). Effects of thresholding on correlation-based image similarity metrics. \textit{Front Neurosci, 9}, 418. \href{https://www.ncbi.nlm.nih.gov/pmc/articles/PMC4625081}{OA} \href{http://dx.doi.org/10.3389/fnins.2015.00418}{DOI} \vspace{2mm}

\subsection*{2014}

Aron AR, Robbins TW, Poldrack RA (2014). Right inferior frontal cortex: addressing the rebuttals. \textit{Front Hum Neurosci, 8}, 905. \href{https://www.ncbi.nlm.nih.gov/pmc/articles/PMC4227507}{OA} \href{http://dx.doi.org/10.3389/fnhum.2014.00905}{DOI} \vspace{2mm}

Aron AR, Robbins TW, Poldrack RA (2014). Inhibition and the right inferior frontal cortex: one decade on. \textit{Trends Cogn. Sci. (Regul. Ed.), 18}, 177-85. \href{http://dx.doi.org/10.1016/j.tics.2013.12.003}{DOI} \vspace{2mm}

Congdon E et al. (2014). Neural activation during response inhibition in adult attention-deficit/hyperactivity disorder: preliminary findings on the effects of medication and symptom severity. \textit{Psychiatry Res, 222}, 17-28. \href{https://www.ncbi.nlm.nih.gov/pmc/articles/PMC4009011}{OA} \href{http://dx.doi.org/10.1016/j.pscychresns.2014.02.002}{DOI} \vspace{2mm}

Davis T, LaRocque KF, Mumford JA, Norman KA, Wagner AD, Poldrack RA (2014). What do differences between multi-voxel and univariate analysis mean? How subject-, voxel-, and trial-level variance impact fMRI analysis. \textit{Neuroimage, 97}, 271-83. \href{https://www.ncbi.nlm.nih.gov/pmc/articles/PMC4115449}{OA} \href{http://dx.doi.org/10.1016/j.neuroimage.2014.04.037}{DOI} \vspace{2mm}

Davis T, Poldrack RA (2014). Quantifying the internal structure of categories using a neural typicality measure. \textit{Cereb. Cortex, 24}, 1720-37. \href{http://dx.doi.org/10.1093/cercor/bht014}{DOI} \vspace{2mm}

Davis T, Xue G, Love BC, Preston AR, Poldrack RA (2014). Global neural pattern similarity as a common basis for categorization and recognition memory. \textit{J. Neurosci., 34}, 7472-84. \href{https://www.ncbi.nlm.nih.gov/pmc/articles/PMC4035513}{OA} \href{http://dx.doi.org/10.1523/jneurosci.3376-13.2014}{DOI} \vspace{2mm}

Hastings J, Frishkoff GA, Smith B, Jensen M, Poldrack RA, Lomax J, Bandrowski A, Imam F, Turner JA, Martone ME (2014). Interdisciplinary perspectives on the development, integration, and application of cognitive ontologies. \textit{Front Neuroinform, 8}, 62. \href{https://www.ncbi.nlm.nih.gov/pmc/articles/PMC4064452}{OA} \href{http://dx.doi.org/10.3389/fninf.2014.00062}{DOI} \vspace{2mm}

Helfinstein SM, Schonberg T, Congdon E, Karlsgodt KH, Mumford JA, Sabb FW, Cannon TD, London ED, Bilder RM, Poldrack RA (2014). Predicting risky choices from brain activity patterns. \textit{Proc. Natl. Acad. Sci. U.S.A., 111}, 2470-5. \href{https://www.ncbi.nlm.nih.gov/pmc/articles/PMC3932884}{OA} \href{https://openneuro.org/datasets/ds000030/versions/1.0.0}{Data} \href{http://dx.doi.org/10.1073/pnas.1321728111}{DOI} \vspace{2mm}

Jimura K, Cazalis F, Stover ER, Poldrack RA (2014). The neural basis of task switching changes with skill acquisition. \textit{Front Hum Neurosci, 8}, 339. \href{https://www.ncbi.nlm.nih.gov/pmc/articles/PMC4033195}{OA} \href{http://dx.doi.org/10.3389/fnhum.2014.00339}{DOI} \vspace{2mm}

Koyejo O, Khanna R, Ghosh J, Poldrack RA (2014). On Prior Distributions and Approximate Inference for Structured Variables. \textit{Advances in Neural Information Processing Systems, 27.0}. \href{http://dx.doi.org/zndnlplo}{DOI} \vspace{2mm}

Mumford JA, Davis T, Poldrack RA (2014). The impact of study design on pattern estimation for single-trial multivariate pattern analysis. \textit{Neuroimage, 103}, 130-138. \href{http://dx.doi.org/10.1016/j.neuroimage.2014.09.026}{DOI} \vspace{2mm}

Poldrack RA, Gorgolewski KJ (2014). Making big data open: data sharing in neuroimaging. \textit{Nat. Neurosci., 17}, 1510-7. \href{http://dx.doi.org/10.1038/nn.3818}{DOI} \vspace{2mm}

Schonberg T, Bakkour A, Hover AM, Mumford JA, Nagar L, Perez J, Poldrack RA (2014). Changing value through cued approach: an automatic mechanism of behavior change. \textit{Nat. Neurosci., 17}, 625-30. \href{https://www.ncbi.nlm.nih.gov/pmc/articles/PMC4041518}{OA} \href{http://dx.doi.org/10.1038/nn.3673}{DOI} \vspace{2mm}

Schonberg T, Bakkour A, Hover AM, Mumford JA, Poldrack RA (2014). Influencing food choices by training: evidence for modulation of frontoparietal control signals. \textit{J Cogn Neurosci, 26}, 247-68. \href{https://www.ncbi.nlm.nih.gov/pmc/articles/PMC4066661}{OA} \href{http://dx.doi.org/10.1162/jocn\_a\_00495}{DOI} \vspace{2mm}

Thakkar KN, Congdon E, Poldrack RA, Sabb FW, London ED, Cannon TD, Bilder RM (2014). Women are more sensitive than men to prior trial events on the Stop-signal task. \textit{Br J Psychol, 105}, 254-72. \href{https://www.ncbi.nlm.nih.gov/pmc/articles/PMC4000536}{OA} \href{http://dx.doi.org/10.1111/bjop.12034}{DOI} \vspace{2mm}

Wagshal D, Knowlton BJ, Cohen JR, Poldrack RA, Bookheimer SY, Bilder RM, Asarnow RF (2014). Impaired automatization of a cognitive skill in first-degree relatives of patients with schizophrenia. \textit{Psychiatry Res, 215}, 294-9. \href{https://www.ncbi.nlm.nih.gov/pmc/articles/PMC4191851}{OA} \href{http://dx.doi.org/10.1016/j.psychres.2013.11.024}{DOI} \vspace{2mm}

Wagshal D, Knowlton BJ, Suthana NA, Cohen JR, Poldrack RA, Bookheimer SY, Bilder RM, Asarnow RF (2014). Evidence for corticostriatal dysfunction during cognitive skill learning in adolescent siblings of patients with childhood-onset schizophrenia. \textit{Schizophr Bull, 40}, 1030-9. \href{https://www.ncbi.nlm.nih.gov/pmc/articles/PMC4133665}{OA} \href{http://dx.doi.org/10.1093/schbul/sbt147}{DOI} \vspace{2mm}

White CN, Congdon E, Mumford JA, Karlsgodt KH, Sabb FW, Freimer NB, London ED, Cannon TD, Bilder RM, Poldrack RA (2014). Decomposing decision components in the stop-signal task: a model-based approach to individual differences in inhibitory control. \textit{J Cogn Neurosci, 26}, 1601-14. \href{https://www.ncbi.nlm.nih.gov/pmc/articles/PMC4119005}{OA} \href{https://openneuro.org/datasets/ds000030/versions/1.0.0}{Data} \href{http://dx.doi.org/10.1162/jocn\_a\_00567}{DOI} \vspace{2mm}

White CN, Poldrack RA (2014). Decomposing bias in different types of simple decisions. \textit{J Exp Psychol Learn Mem Cogn, 40}, 385-398. \href{http://dx.doi.org/10.1037/a0034851}{DOI} \vspace{2mm}

\subsection*{2013}

Barch DM et al. (2013). Function in the human connectome: task-fMRI and individual differences in behavior. \textit{Neuroimage, 80}, 169-89. \href{https://www.ncbi.nlm.nih.gov/pmc/articles/PMC4011498}{OA} \href{http://dx.doi.org/10.1016/j.neuroimage.2013.05.033}{DOI} \vspace{2mm}

Brakewood B, Poldrack RA (2013). The ethics of secondary data analysis: considering the application of Belmont principles to the sharing of neuroimaging data. \textit{Neuroimage, 82}, 671-6. \href{http://dx.doi.org/10.1016/j.neuroimage.2013.02.040}{DOI} \vspace{2mm}

Congdon E, Bato AA, Schonberg T, Mumford JA, Karlsgodt KH, Sabb FW, London ED, Cannon TD, Bilder RM, Poldrack RA (2013). Differences in neural activation as a function of risk-taking task parameters. \textit{Front Neurosci, 7}, 173. \href{https://www.ncbi.nlm.nih.gov/pmc/articles/PMC3786224}{OA} \href{http://dx.doi.org/10.3389/fnins.2013.00173}{DOI} \vspace{2mm}

Davis T, Poldrack RA (2013). Measuring neural representations with fMRI: practices and pitfalls. \textit{Ann. N. Y. Acad. Sci., 1296}, 108-34. \href{http://dx.doi.org/10.1111/nyas.12156}{DOI} \vspace{2mm}

Fox CR, Poldrack RA (2013). Prospect theory and the brain.. In \textit{Neuroeconomics: Decision Making and the Brain (2nd Edition).} Elsevier. \href{http://dx.doi.org/9780124160088}{DOI} \vspace{2mm}

Galván A, Schonberg T, Mumford J, Kohno M, Poldrack RA, London ED (2013). Greater risk sensitivity of dorsolateral prefrontal cortex in young smokers than in nonsmokers. \textit{Psychopharmacology (Berl.), 229}, 345-55. \href{https://www.ncbi.nlm.nih.gov/pmc/articles/PMC3758460}{OA} \href{http://dx.doi.org/10.1007/s00213-013-3113-x}{DOI} \vspace{2mm}

Hsieh CJ, Sustik MA, Dhillon IS, Ravikumar R,  Poldrack RA (2013). BIG \& QUIC: Sparse Inverse Covariance Estimation for a Million Variables. \textit{Advances in Neural Information Processing Systems, }. \href{http://dx.doi.org/cxuijsrb}{DOI} \vspace{2mm}

Koyejo O, Patel P, Ghosh J, Poldrack RA (2013). Learning Predictive Cognitive Structure from fMRI Using Supervised Topic Models \textit{2013 International Workshop on Pattern Recognition in Neuroimaging}. \href{http://dx.doi.org/10.1109/prni.2013.12}{DOI} \vspace{2mm}

Malecek NJ, Poldrack RA (2013). Beyond dopamine: the noradrenergic system and mental effort. \textit{Behav Brain Sci, 36}, 698-9; discussion 707-26. \href{http://dx.doi.org/10.1017/s0140525x13001106}{DOI} \vspace{2mm}

Park M, Koyejo O, Ghosh J, Poldrack RA, Pillow JW (2013). Bayesian structure learning for functional neuroimaging. \textit{Sixteenth International Conference on Artificial Intelligence and Statistics (AIST ATS)., }. \href{http://dx.doi.org/htypsryf}{DOI} \vspace{2mm}

Poldrack RA, Barch DM, Mitchell JP, Wager TD, Wagner AD, Devlin JT, Cumba C, Koyejo O, Milham MP (2013). Toward open sharing of task-based fMRI data: the OpenfMRI project. \textit{Front Neuroinform, 7}, 12. \href{https://www.ncbi.nlm.nih.gov/pmc/articles/PMC3703526}{OA} \href{http://dx.doi.org/10.3389/fninf.2013.00012}{DOI} \vspace{2mm}

Poline JB, Poldrack RA (2013). Introduction to the special issue: toward a new era of databasing and data sharing for neuroimaging. \textit{Neuroimage, 82}, 645-6. \href{http://dx.doi.org/10.1016/j.neuroimage.2013.08.044}{DOI} \vspace{2mm}

Van Essen DC, Smith SM, Barch DM, Behrens TE, Yacoub E, Ugurbil K (2013). The WU-Minn Human Connectome Project: an overview. \textit{Neuroimage, 80}, 62-79. \href{https://www.ncbi.nlm.nih.gov/pmc/articles/PMC3724347}{OA} \href{http://dx.doi.org/10.1016/j.neuroimage.2013.05.041}{DOI} \vspace{2mm}

White CN, Poldrack RA (2013). Using fMRI to Constrain Theories of Cognition. \textit{Perspect Psychol Sci, 8}, 79-83. \href{http://dx.doi.org/10.1177/1745691612469029}{DOI} \vspace{2mm}

Xue G, Dong Q, Chen C, Lu ZL, Mumford JA, Poldrack RA (2013). Complementary role of frontoparietal activity and cortical pattern similarity in successful episodic memory encoding. \textit{Cereb. Cortex, 23}, 1562-71. \href{https://www.ncbi.nlm.nih.gov/pmc/articles/PMC3726068}{OA} \href{http://dx.doi.org/10.1093/cercor/bhs143}{DOI} \vspace{2mm}

\subsection*{2012}

Congdon E, Mumford JA, Cohen JR, Galvan A, Canli T, Poldrack RA (2012). Measurement and reliability of response inhibition. \textit{Front Psychol, 3}, 37. \href{https://www.ncbi.nlm.nih.gov/pmc/articles/PMC3283117}{OA} \href{http://dx.doi.org/10.3389/fpsyg.2012.00037}{DOI} \vspace{2mm}

Courtney KE, Arellano R, Barkley-Levenson E, Gálvan A, Poldrack RA, Mackillop J, Jentsch JD, Ray LA (2012). The relationship between measures of impulsivity and alcohol misuse: an integrative structural equation modeling approach. \textit{Alcohol. Clin. Exp. Res., 36}, 923-31. \href{https://www.ncbi.nlm.nih.gov/pmc/articles/PMC3291799}{OA} \href{http://dx.doi.org/10.1111/j.1530-0277.2011.01635.x}{DOI} \vspace{2mm}

Ghahremani DG et al. (2012). Striatal dopamine D₂/D₃ receptors mediate response inhibition and related activity in frontostriatal neural circuitry in humans. \textit{J. Neurosci., 32}, 7316-24. \href{https://www.ncbi.nlm.nih.gov/pmc/articles/PMC3517177}{OA} \href{http://dx.doi.org/10.1523/jneurosci.4284-11.2012}{DOI} \vspace{2mm}

Helfinstein SM, Poldrack RA (2012). The young and the reckless. \textit{Nat. Neurosci., 15}, 803-5. \href{http://dx.doi.org/10.1038/nn.3116}{DOI} \vspace{2mm}

Jimura K, Poldrack RA (2012). Analyses of regional-average activation and multivoxel pattern information tell complementary stories. \textit{Neuropsychologia, 50}, 544-52. \href{http://dx.doi.org/10.1016/j.neuropsychologia.2011.11.007}{DOI} \vspace{2mm}

Mumford JA, Turner BO, Ashby FG, Poldrack RA (2012). Deconvolving BOLD activation in event-related designs for multivoxel pattern classification analyses. \textit{Neuroimage, 59}, 2636-43. \href{https://www.ncbi.nlm.nih.gov/pmc/articles/PMC3251697}{OA} \href{http://dx.doi.org/10.1016/j.neuroimage.2011.08.076}{DOI} \vspace{2mm}

Poldrack RA (2012). The future of fMRI in cognitive neuroscience. \textit{Neuroimage, 62}, 1216-20. \href{https://www.ncbi.nlm.nih.gov/pmc/articles/PMC4131441}{OA} \href{http://dx.doi.org/10.1016/j.neuroimage.2011.08.007}{DOI} \vspace{2mm}

Poldrack RA, Mumford JA, Schonberg T, Kalar D, Barman B, Yarkoni T (2012). Discovering relations between mind, brain, and mental disorders using topic mapping. \textit{PLoS Comput. Biol., 8}, e1002707. \href{https://www.ncbi.nlm.nih.gov/pmc/articles/PMC3469446}{OA} \href{https://github.com/poldrack/LatentStructure}{Code} \href{http://dx.doi.org/10.1371/journal.pcbi.1002707}{DOI} \vspace{2mm}

Poline JB et al. (2012). Data sharing in neuroimaging research. \textit{Front Neuroinform, 6}, 9. \href{https://www.ncbi.nlm.nih.gov/pmc/articles/PMC3319918}{OA} \href{http://dx.doi.org/10.3389/fninf.2012.00009}{DOI} \vspace{2mm}

Poline JB, Poldrack RA (2012). Frontiers in brain imaging methods grand challenge. \textit{Front Neurosci, 6}, 96. \href{https://www.ncbi.nlm.nih.gov/pmc/articles/PMC3390895}{OA} \href{http://dx.doi.org/10.3389/fnins.2012.00096}{DOI} \vspace{2mm}

Satpute AB, Mumford JA, Naliboff BD, Poldrack RA (2012). Human anterior and posterior hippocampus respond distinctly to state and trait anxiety. \textit{Emotion, 12}, 58-68. \href{http://dx.doi.org/10.1037/a0026517}{DOI} \vspace{2mm}

Schonberg T, Fox CR, Mumford JA, Congdon E, Trepel C, Poldrack RA (2012). Decreasing ventromedial prefrontal cortex activity during sequential risk-taking: an FMRI investigation of the balloon analog risk task. \textit{Front Neurosci, 6}, 80. \href{https://www.ncbi.nlm.nih.gov/pmc/articles/PMC3366349}{OA} \href{https://openneuro.org/datasets/ds000001/versions/1.0.0}{Data} \href{http://dx.doi.org/10.3389/fnins.2012.00080}{DOI} \vspace{2mm}

Turner BO, Mumford JA, Poldrack RA, Ashby FG (2012). Spatiotemporal activity estimation for multivoxel pattern analysis with rapid event-related designs. \textit{Neuroimage, 62}, 1429-38. \href{https://www.ncbi.nlm.nih.gov/pmc/articles/PMC3408801}{OA} \href{http://dx.doi.org/10.1016/j.neuroimage.2012.05.057}{DOI} \vspace{2mm}

Wagshal D, Knowlton BJ, Cohen JR, Poldrack RA, Bookheimer SY, Bilder RM, Fernandez VG, Asarnow RF (2012). Deficits in probabilistic classification learning and liability for schizophrenia. \textit{Psychiatry Res, 200}, 167-72. \href{https://www.ncbi.nlm.nih.gov/pmc/articles/PMC5332149}{OA} \href{http://dx.doi.org/10.1016/j.psychres.2012.06.009}{DOI} \vspace{2mm}

White CN, Mumford JA, Poldrack RA (2012). Perceptual criteria in the human brain. \textit{J. Neurosci., 32}, 16716-24. \href{https://www.ncbi.nlm.nih.gov/pmc/articles/PMC6621768}{OA} \href{http://dx.doi.org/10.1523/jneurosci.1744-12.2012}{DOI} \vspace{2mm}

\subsection*{2011}

Cohen JR, Asarnow RF, Sabb FW, Bilder RM, Bookheimer SY, Knowlton BJ, Poldrack RA (2011). Decoding continuous variables from neuroimaging data: basic and clinical applications. \textit{Front Neurosci, 5}, 75. \href{https://www.ncbi.nlm.nih.gov/pmc/articles/PMC3118657}{OA} \href{http://dx.doi.org/10.3389/fnins.2011.00075}{DOI} \vspace{2mm}

Galván A, Poldrack RA, Baker CM, McGlennen KM, London ED (2011). Neural correlates of response inhibition and cigarette smoking in late adolescence. \textit{Neuropsychopharmacology, 36}, 970-8. \href{https://www.ncbi.nlm.nih.gov/pmc/articles/PMC3077266}{OA} \href{http://dx.doi.org/10.1038/npp.2010.235}{DOI} \vspace{2mm}

Ghahremani DG, Tabibnia G, Monterosso J, Hellemann G, Poldrack RA, London ED (2011). Effect of modafinil on learning and task-related brain activity in methamphetamine-dependent and healthy individuals. \textit{Neuropsychopharmacology, 36}, 950-9. \href{https://www.ncbi.nlm.nih.gov/pmc/articles/PMC3077264}{OA} \href{http://dx.doi.org/10.1038/npp.2010.233}{DOI} \vspace{2mm}

Hübner NO, Assadian O, Poldrack R, Duty O, Schwarzer H, Möller H, Kober P, Räther M, Schröder LW, Sinha J, Lerch MM, Kramer A (2011). Endowashers: an overlooked risk for possible post-endoscopic infections. \textit{GMS Krankenhhyg Interdiszip, 6}, Doc13. \href{https://www.ncbi.nlm.nih.gov/pmc/articles/PMC3252647}{OA} \href{http://dx.doi.org/10.3205/dgkh000170}{DOI} \vspace{2mm}

Lenartowicz A, Verbruggen F, Logan GD, Poldrack RA (2011). Inhibition-related activation in the right inferior frontal gyrus in the absence of inhibitory cues. \textit{J Cogn Neurosci, 23}, 3388-99. \href{http://dx.doi.org/10.1162/jocn\_a\_00031}{DOI} \vspace{2mm}

Poldrack RA (2011). Inferring mental states from neuroimaging data: from reverse inference to large-scale decoding. \textit{Neuron, 72}, 692-7. \href{https://www.ncbi.nlm.nih.gov/pmc/articles/PMC3240863}{OA} \href{http://dx.doi.org/10.1016/j.neuron.2011.11.001}{DOI} \vspace{2mm}

Poldrack RA, Kittur A, Kalar D, Miller E, Seppa C, Gil Y, Parker DS, Sabb FW, Bilder RM (2011). The cognitive atlas: toward a knowledge foundation for cognitive neuroscience. \textit{Front Neuroinform, 5}, 17. \href{https://www.ncbi.nlm.nih.gov/pmc/articles/PMC3167196}{OA} \href{http://dx.doi.org/10.3389/fninf.2011.00017}{DOI} \vspace{2mm}

Poldrack RA, Nichols T, Mumford J (2011).  \textit{Handbook of Functional MRI Data Analysis}. Cambridge University Press. \vspace{2mm}

Rizk-Jackson A, Stoffers D, Sheldon S, Kuperman J, Dale A, Goldstein J, Corey-Bloom J, Poldrack RA, Aron AR (2011). Evaluating imaging biomarkers for neurodegeneration in pre-symptomatic Huntington's disease using machine learning techniques. \textit{Neuroimage, 56}, 788-96. \href{http://dx.doi.org/10.1016/j.neuroimage.2010.04.273}{DOI} \vspace{2mm}

Schonberg T, Fox CR, Poldrack RA (2011). Mind the gap: bridging economic and naturalistic risk-taking with cognitive neuroscience. \textit{Trends Cogn. Sci. (Regul. Ed.), 15}, 11-9. \href{https://www.ncbi.nlm.nih.gov/pmc/articles/PMC3014440}{OA} \href{http://dx.doi.org/10.1016/j.tics.2010.10.002}{DOI} \vspace{2mm}

Stern JM, Caporro M, Haneef Z, Yeh HJ, Buttinelli C, Lenartowicz A, Mumford JA, Parvizi J, Poldrack RA (2011). Functional imaging of sleep vertex sharp transients. \textit{Clin Neurophysiol, 122}, 1382-6. \href{https://www.ncbi.nlm.nih.gov/pmc/articles/PMC3105179}{OA} \href{http://dx.doi.org/10.1016/j.clinph.2010.12.049}{DOI} \vspace{2mm}

Tabibnia G, Monterosso JR, Baicy K, Aron AR, Poldrack RA, Chakrapani S, Lee B, London ED (2011). Different forms of self-control share a neurocognitive substrate. \textit{J. Neurosci., 31}, 4805-10. \href{https://www.ncbi.nlm.nih.gov/pmc/articles/PMC3096483}{OA} \href{http://dx.doi.org/10.1523/jneurosci.2859-10.2011}{DOI} \vspace{2mm}

Xue G, Mei L, Chen C, Lu ZL, Poldrack R, Dong Q (2011). Spaced learning enhances subsequent recognition memory by reducing neural repetition suppression. \textit{J Cogn Neurosci, 23}, 1624-33. \href{https://www.ncbi.nlm.nih.gov/pmc/articles/PMC3297428}{OA} \href{http://dx.doi.org/10.1162/jocn.2010.21532}{DOI} \vspace{2mm}

Yarkoni T, Poldrack RA, Nichols TE, Van Essen DC, Wager TD (2011). Large-scale automated synthesis of human functional neuroimaging data. \textit{Nat. Methods, 8}, 665-70. \href{https://www.ncbi.nlm.nih.gov/pmc/articles/PMC3146590}{OA} \href{http://dx.doi.org/10.1038/nmeth.1635}{DOI} \vspace{2mm}

\subsection*{2010}

Cho S, Moody TD, Fernandino L, Mumford JA, Poldrack RA, Cannon TD, Knowlton BJ, Holyoak KJ (2010). Common and dissociable prefrontal loci associated with component mechanisms of analogical reasoning. \textit{Cereb. Cortex, 20}, 524-33. \href{http://dx.doi.org/10.1093/cercor/bhp121}{DOI} \vspace{2mm}

Cohen JR, Asarnow RF, Sabb FW, Bilder RM, Bookheimer SY, Knowlton BJ, Poldrack RA (2010). Decoding developmental differences and individual variability in response inhibition through predictive analyses across individuals. \textit{Front Hum Neurosci, 4}, 47. \href{https://www.ncbi.nlm.nih.gov/pmc/articles/PMC2906202}{OA} \href{http://dx.doi.org/10.3389/fnhum.2010.00047}{DOI} \vspace{2mm}

Cohen JR, Asarnow RF, Sabb FW, Bilder RM, Bookheimer SY, Knowlton BJ, Poldrack RA (2010). A unique adolescent response to reward prediction errors. \textit{Nat. Neurosci., 13}, 669-71. \href{https://www.ncbi.nlm.nih.gov/pmc/articles/PMC2876211}{OA} \href{http://dx.doi.org/10.1038/nn.2558}{DOI} \vspace{2mm}

Congdon E, Mumford JA, Cohen JR, Galvan A, Aron AR, Xue G, Miller E, Poldrack RA (2010). Engagement of large-scale networks is related to individual differences in inhibitory control. \textit{Neuroimage, 53}, 653-63. \href{https://www.ncbi.nlm.nih.gov/pmc/articles/PMC2930099}{OA} \href{http://dx.doi.org/10.1016/j.neuroimage.2010.06.062}{DOI} \vspace{2mm}

Congdon E, Poldrack RA, Freimer NB (2010). Neurocognitive phenotypes and genetic dissection of disorders of brain and behavior. \textit{Neuron, 68}, 218-30. \href{https://www.ncbi.nlm.nih.gov/pmc/articles/PMC4123421}{OA} \href{http://dx.doi.org/10.1016/j.neuron.2010.10.007}{DOI} \vspace{2mm}

Crone EA, Poldrack RA, Durston S (2010). Challenges and methods in developmental neuroimaging. \textit{Hum Brain Mapp, 31}, 835-7. \href{http://dx.doi.org/10.1002/hbm.21053}{DOI} \vspace{2mm}

Ghahremani DG, Monterosso J, Jentsch JD, Bilder RM, Poldrack RA (2010). Neural components underlying behavioral flexibility in human reversal learning. \textit{Cereb. Cortex, 20}, 1843-52. \href{https://www.ncbi.nlm.nih.gov/pmc/articles/PMC2901019}{OA} \href{http://dx.doi.org/10.1093/cercor/bhp247}{DOI} \vspace{2mm}

Kenner NM, Mumford JA, Hommer RE, Skup M, Leibenluft E, Poldrack RA (2010). Inhibitory motor control in response stopping and response switching. \textit{J. Neurosci., 30}, 8512-8. \href{https://www.ncbi.nlm.nih.gov/pmc/articles/PMC2905623}{OA} \href{http://dx.doi.org/10.1523/jneurosci.1096-10.2010}{DOI} \vspace{2mm}

Kriegeskorte N, Lindquist MA, Nichols TE, Poldrack RA, Vul E (2010). Everything you never wanted to know about circular analysis, but were afraid to ask. \textit{J. Cereb. Blood Flow Metab., 30}, 1551-7. \href{https://www.ncbi.nlm.nih.gov/pmc/articles/PMC2949251}{OA} \href{http://dx.doi.org/10.1038/jcbfm.2010.86}{DOI} \vspace{2mm}

Lenartowicz A, Kalar DJ, Congdon E, Poldrack RA (2010). Towards an ontology of cognitive control. \textit{Top Cogn Sci, 2}, 678-92. \href{http://dx.doi.org/10.1111/j.1756-8765.2010.01100.x}{DOI} \vspace{2mm}

Lenartowicz A, Poldrack R (2010). Brain Imaging. In \textit{Encyclopedia of Behavioral Neuroscience.} (p. 187-193). Elsevier. \href{http://dx.doi.org/10.1016/b978-0-08-045396-5.00052-x}{DOI} \vspace{2mm}

Mumford JA, Horvath S, Oldham MC, Langfelder P, Geschwind DH, Poldrack RA (2010). Detecting network modules in fMRI time series: a weighted network analysis approach. \textit{Neuroimage, 52}, 1465-76. \href{https://www.ncbi.nlm.nih.gov/pmc/articles/PMC3632300}{OA} \href{http://dx.doi.org/10.1016/j.neuroimage.2010.05.047}{DOI} \vspace{2mm}

Poldrack RA (2010). Mapping Mental Function to Brain Structure: How Can Cognitive Neuroimaging Succeed? \textit{Perspect Psychol Sci, 5}, 753-61. \href{https://www.ncbi.nlm.nih.gov/pmc/articles/PMC4112478}{OA} \href{http://dx.doi.org/10.1177/1745691610388777}{DOI} \vspace{2mm}

Poldrack RA (2010). Subtraction and beyond: The logic of experimental designs for neuroimaging.. In \textit{Foundational Issues in Human Brain Mapping.} (p. 147-160). Cambridge, MA: MIT Press. \href{http://dx.doi.org/10.7551/mitpress/9780262014021.001.0001}{DOI} \vspace{2mm}

Poldrack RA (2010). Interpreting developmental changes in neuroimaging signals. \textit{Hum Brain Mapp, 31}, 872-8. \href{http://dx.doi.org/10.1002/hbm.21039}{DOI} \vspace{2mm}

Poldrack RA, Carr V, Foerde K (2010). Flexibility and generalization in memory systems.. In \textit{Generalization of Knowledge: Multidisciplinary perspectives.} (p. 53-70). New York, NY: Psychology Press.. \href{http://dx.doi.org/9781136945465}{DOI} \vspace{2mm}

Poldrack RA, Mumford JA (2010). On the proper role of non-independent ROI analysis: A commentary on Vul and Kanwisher.. In \textit{Foundational Issues in Human Brain Mapping.} (p. 93-96). Cambridge, MA: MIT Press. \href{http://dx.doi.org/10.7551/mitpress/9780262014021.001.0001-ymwkkriy}{DOI} \vspace{2mm}

Ramsey JD, Hanson SJ, Hanson C, Halchenko YO, Poldrack RA, Glymour C (2010). Six problems for causal inference from fMRI. \textit{Neuroimage, 49}, 1545-58. \href{http://dx.doi.org/10.1016/j.neuroimage.2009.08.065}{DOI} \vspace{2mm}

Scott-Van Zeeland AA et al. (2010). Altered functional connectivity in frontal lobe circuits is associated with variation in the autism risk gene CNTNAP2. \textit{Sci Transl Med, 2}, 56ra80. \href{https://www.ncbi.nlm.nih.gov/pmc/articles/PMC3065863}{OA} \href{http://dx.doi.org/10.1126/scitranslmed.3001344}{DOI} \vspace{2mm}

Scott-Van Zeeland AA, Dapretto M, Ghahremani DG, Poldrack RA, Bookheimer SY (2010). Reward processing in autism. \textit{Autism Res, 3}, 53-67. \href{https://www.ncbi.nlm.nih.gov/pmc/articles/PMC3076289}{OA} \href{http://dx.doi.org/10.1002/aur.122}{DOI} \vspace{2mm}

Xue G, Dong Q, Chen C, Lu Z, Mumford JA, Poldrack RA (2010). Greater neural pattern similarity across repetitions is associated with better memory. \textit{Science, 330}, 97-101. \href{https://www.ncbi.nlm.nih.gov/pmc/articles/PMC2952039}{OA} \href{http://dx.doi.org/10.1126/science.1193125}{DOI} \vspace{2mm}

Xue G, Mei L, Chen C, Lu ZL, Poldrack RA, Dong Q (2010). Facilitating memory for novel characters by reducing neural repetition suppression in the left fusiform cortex. \textit{PLoS ONE, 5}, e13204. \href{https://www.ncbi.nlm.nih.gov/pmc/articles/PMC2950859}{OA} \href{http://dx.doi.org/10.1371/journal.pone.0013204}{DOI} \vspace{2mm}

Yarkoni T, Poldrack RA, Van Essen DC, Wager TD (2010). Cognitive neuroscience 2.0: building a cumulative science of human brain function. \textit{Trends Cogn. Sci. (Regul. Ed.), 14}, 489-96. \href{https://www.ncbi.nlm.nih.gov/pmc/articles/PMC2963679}{OA} \href{http://dx.doi.org/10.1016/j.tics.2010.08.004}{DOI} \vspace{2mm}

\subsection*{2009}

Aron AR, Wise SP, Poldrack RA (2009). Role of the basal ganglia in cognition.. In \textit{The New Encyclopedia of Neuroscience (Volume 2).} (p. 1069-1077). Oxford, UK: Academic Press. \href{http://dx.doi.org/978-0080447971}{DOI} \vspace{2mm}

Barch DM, Braver TS, Carter CS, Poldrack RA, Robbins TW (2009). CNTRICS final task selection: executive control. \textit{Schizophr Bull, 35}, 115-35. \href{https://www.ncbi.nlm.nih.gov/pmc/articles/PMC2643948}{OA} \href{http://dx.doi.org/10.1093/schbul/sbn154}{DOI} \vspace{2mm}

Bilder RM, Poldrack RA, Parker DS, Reise SP, Jentsch JD, Cannon T, London E, Sabb FW, Foland L, Rizk-Jackson A, Kalar D, Brown N, Carstensen A, Freimer N (2009). Cognitive phenomics.. In \textit{Handbook of Neuropsychology of Mental Disorders.} Cambridge: Cambridge University Press. \href{http://dx.doi.org/rjoyrcvt}{DOI} \vspace{2mm}

Bilder RM, Sabb FW, Cannon TD, London ED, Jentsch JD, Parker DS, Poldrack RA, Evans C, Freimer NB (2009). Phenomics: the systematic study of phenotypes on a genome-wide scale. \textit{Neuroscience, 164}, 30-42. \href{https://www.ncbi.nlm.nih.gov/pmc/articles/PMC2760679}{OA} \href{http://dx.doi.org/10.1016/j.neuroscience.2009.01.027}{DOI} \vspace{2mm}

Bilder RM, Sabb FW, Parker DS, Kalar D, Chu WW, Fox J, Freimer NB, Poldrack RA (2009). Cognitive ontologies for neuropsychiatric phenomics research. \textit{Cogn Neuropsychiatry, 14}, 419-50. \href{https://www.ncbi.nlm.nih.gov/pmc/articles/PMC2752634}{OA} \href{http://dx.doi.org/10.1080/13546800902787180}{DOI} \vspace{2mm}

Foerde K, Poldrack RA (2009). Procedural learning in humans.. In \textit{The New Encyclopedia of Neuroscience.} (p. 1083-1091). Oxford, UK: Academic Press. \href{http://dx.doi.org/978-0080447971-xey}{DOI} \vspace{2mm}

Fox CR, Poldrack RA (2009). Prospect Theory and the Brain. In \textit{Neuroeconomics.} (p. 145-173). Elsevier. \href{http://dx.doi.org/10.1016/b978-0-12-374176-9.00011-7}{DOI} \vspace{2mm}

Ghahremani DG, Poldrack RA (2009). Neuroimaging and interactive memory systems. In \textit{Neuroimaging of Human MemoryLinking cognitive processes to neural systems.} (p. 77-88). Oxford University Press. \href{http://dx.doi.org/10.1093/acprof:oso/9780199217298.003.0006}{DOI} \vspace{2mm}

Glimcher P, Fehr E, Camerer C, Poldrack RA (Eds.) (2009).  \textit{Handbook of Neuroeconomics}. San Diego: Academic Press. \vspace{2mm}

Glimcher PW, Camerer CF, Fehr E, Poldrack RA (2009). Introduction. In \textit{Neuroeconomics.} (p. 1-12). Elsevier. \href{http://dx.doi.org/10.1016/b978-0-12-374176-9.00001-4}{DOI} \vspace{2mm}

Lee B et al. (2009). Striatal dopamine d2/d3 receptor availability is reduced in methamphetamine dependence and is linked to impulsivity. \textit{J. Neurosci., 29}, 14734-40. \href{https://www.ncbi.nlm.nih.gov/pmc/articles/PMC2822639}{OA} \href{http://dx.doi.org/10.1523/jneurosci.3765-09.2009}{DOI} \vspace{2mm}

Poldrack RA, Halchenko YO, Hanson SJ (2009). Decoding the large-scale structure of brain function by classifying mental States across individuals. \textit{Psychol Sci, 20}, 1364-72. \href{https://www.ncbi.nlm.nih.gov/pmc/articles/PMC2935493}{OA} \href{http://dx.doi.org/10.1111/j.1467-9280.2009.02460.x}{DOI} \vspace{2mm}

Poldrack RA, Mumford JA (2009). Independence in ROI analysis: where is the voodoo? \textit{Soc Cogn Affect Neurosci, 4}, 208-13. \href{https://www.ncbi.nlm.nih.gov/pmc/articles/PMC2686233}{OA} \href{http://dx.doi.org/10.1093/scan/nsp011}{DOI} \vspace{2mm}

Sabb FW et al. (2009). Challenges in phenotype definition in the whole-genome era: multivariate models of memory and intelligence. \textit{Neuroscience, 164}, 88-107. \href{https://www.ncbi.nlm.nih.gov/pmc/articles/PMC2766544}{OA} \href{http://dx.doi.org/10.1016/j.neuroscience.2009.05.013}{DOI} \vspace{2mm}

Thompson PM, Miller MI, Poldrack RA, Nichols TE, Taylor JE, Worsley KJ, Ratnanather JT (2009). Special issue on mathematics in brain imaging. \textit{Neuroimage, 45}, S1-2. \href{http://dx.doi.org/10.1016/j.neuroimage.2008.10.033}{DOI} \vspace{2mm}

Van Horn JD, Poldrack RA (2009). Functional MRI at the crossroads. \textit{Int J Psychophysiol, 73}, 3-9. \href{https://www.ncbi.nlm.nih.gov/pmc/articles/PMC2747289}{OA} \href{http://dx.doi.org/10.1016/j.ijpsycho.2008.11.003}{DOI} \vspace{2mm}

\subsection*{2008}

Cohen JR, Poldrack RA (2008). Automaticity in motor sequence learning does not impair response inhibition. \textit{Psychon Bull Rev, 15}, 108-15. \href{http://dx.doi.org/10.3758/pbr.15.1.108}{DOI} \vspace{2mm}

Foerde K et al. (2008). Selective corticostriatal dysfunction in schizophrenia: examination of motor and cognitive skill learning. \textit{Neuropsychology, 22}, 100-9. \href{http://dx.doi.org/10.1037/0894-4105.22.1.100}{DOI} \vspace{2mm}

Gluck MA, Poldrack RA, Kéri S (2008). The cognitive neuroscience of category learning. \textit{Neurosci Biobehav Rev, 32}, 193-6. \href{http://dx.doi.org/10.1016/j.neubiorev.2007.11.002}{DOI} \vspace{2mm}

Karlsgodt KH, van Erp TG, Poldrack RA, Bearden CE, Nuechterlein KH, Cannon TD (2008). Diffusion tensor imaging of the superior longitudinal fasciculus and working memory in recent-onset schizophrenia. \textit{Biol. Psychiatry, 63}, 512-8. \href{http://dx.doi.org/10.1016/j.biopsych.2007.06.017}{DOI} \vspace{2mm}

Poldrack RA (2008). The role of fMRI in cognitive neuroscience: where do we stand? \textit{Curr. Opin. Neurobiol., 18}, 223-7. \href{http://dx.doi.org/10.1016/j.conb.2008.07.006}{DOI} \vspace{2mm}

Poldrack RA, Fletcher PC, Henson RN, Worsley KJ, Brett M, Nichols TE (2008). Guidelines for reporting an fMRI study. \textit{Neuroimage, 40}, 409-414. \href{https://www.ncbi.nlm.nih.gov/pmc/articles/PMC2287206}{OA} \href{http://dx.doi.org/10.1016/j.neuroimage.2007.11.048}{DOI} \vspace{2mm}

Poldrack RA, Foerde K (2008). Category learning and the memory systems debate. \textit{Neurosci Biobehav Rev, 32}, 197-205. \href{http://dx.doi.org/10.1016/j.neubiorev.2007.07.007}{DOI} \vspace{2mm}

Poldrack RA, Wagner AD (2008). The Interface Between Neuroscience and Psychological Science \textit{Current Directions in Psychological Science, 17}, 61-61. \href{http://dx.doi.org/10.1111/j.1467-8721.2008.00549.x}{DOI} \vspace{2mm}

Shattuck DW, Mirza M, Adisetiyo V, Hojatkashani C, Salamon G, Narr KL, Poldrack RA, Bilder RM, Toga AW (2008). Construction of a 3D probabilistic atlas of human cortical structures. \textit{Neuroimage, 39}, 1064-80. \href{https://www.ncbi.nlm.nih.gov/pmc/articles/PMC2757616}{OA} \href{http://dx.doi.org/10.1016/j.neuroimage.2007.09.031}{DOI} \vspace{2mm}

Tohka J, Foerde K, Aron AR, Tom SM, Toga AW, Poldrack RA (2008). Automatic independent component labeling for artifact removal in fMRI. \textit{Neuroimage, 39}, 1227-45. \href{https://www.ncbi.nlm.nih.gov/pmc/articles/PMC2374836}{OA} \href{http://dx.doi.org/10.1016/j.neuroimage.2007.10.013}{DOI} \vspace{2mm}

Xue G, Aron AR, Poldrack RA (2008). Common neural substrates for inhibition of spoken and manual responses. \textit{Cereb. Cortex, 18}, 1923-32. \href{https://openneuro.org/datasets/ds000007/versions/00001}{Data} \href{http://dx.doi.org/10.1093/cercor/bhm220}{DOI} \vspace{2mm}

Xue G, Ghahremani DG, Poldrack RA (2008). Neural substrates for reversing stimulus-outcome and stimulus-response associations. \textit{J. Neurosci., 28}, 11196-204. \href{https://www.ncbi.nlm.nih.gov/pmc/articles/PMC6671509}{OA} \href{http://dx.doi.org/10.1523/jneurosci.4001-08.2008}{DOI} \vspace{2mm}

\subsection*{2007}

Aron AR, Behrens TE, Smith S, Frank MJ, Poldrack RA (2007). Triangulating a cognitive control network using diffusion-weighted magnetic resonance imaging (MRI) and functional MRI. \textit{J. Neurosci., 27}, 3743-52. \href{https://www.ncbi.nlm.nih.gov/pmc/articles/PMC6672420}{OA} \href{http://dx.doi.org/10.1523/jneurosci.0519-07.2007}{DOI} \vspace{2mm}

Devlin JT, Poldrack RA (2007). In praise of tedious anatomy. \textit{Neuroimage, 37}, 1033-41; discussion 1050-8. \href{https://www.ncbi.nlm.nih.gov/pmc/articles/PMC1986635}{OA} \href{http://dx.doi.org/10.1016/j.neuroimage.2006.09.055}{DOI} \vspace{2mm}

Foerde K, Poldrack RA, Knowlton BJ (2007). Secondary-task effects on classification learning. \textit{Mem Cognit, 35}, 864-74. \href{http://dx.doi.org/10.3758/bf03193461}{DOI} \vspace{2mm}

Mumford JA, Poldrack RA (2007). Modeling group fMRI data. \textit{Soc Cogn Affect Neurosci, 2}, 251-7. \href{https://www.ncbi.nlm.nih.gov/pmc/articles/PMC2569805}{OA} \href{http://dx.doi.org/10.1093/scan/nsm019}{DOI} \vspace{2mm}

Poldrack RA (2007). Region of interest analysis for fMRI. \textit{Soc Cogn Affect Neurosci, 2}, 67-70. \href{https://www.ncbi.nlm.nih.gov/pmc/articles/PMC2555436}{OA} \href{http://dx.doi.org/10.1093/scan/nsm006}{DOI} \vspace{2mm}

Raizada RD, Poldrack RA (2007). Challenge-driven attention: interacting frontal and brainstem systems. \textit{Front Hum Neurosci, 1}, 3. \href{https://www.ncbi.nlm.nih.gov/pmc/articles/PMC2525983}{OA} \href{http://dx.doi.org/10.3389/neuro.09.003.2007}{DOI} \vspace{2mm}

Raizada RD, Poldrack RA (2007). Selective amplification of stimulus differences during categorical processing of speech. \textit{Neuron, 56}, 726-40. \href{http://dx.doi.org/10.1016/j.neuron.2007.11.001}{DOI} \vspace{2mm}

Thermenos HW, Seidman LJ, Poldrack RA, Peace NK, Koch JK, Faraone SV, Tsuang MT (2007). Elaborative verbal encoding and altered anterior parahippocampal activation in adolescents and young adults at genetic risk for schizophrenia using FMRI. \textit{Biol. Psychiatry, 61}, 564-74. \href{http://dx.doi.org/10.1016/j.biopsych.2006.04.044}{DOI} \vspace{2mm}

Tom SM, Fox CR, Trepel C, Poldrack RA (2007). The neural basis of loss aversion in decision-making under risk. \textit{Science, 315}, 515-8. \href{https://openneuro.org/datasets/ds000008/versions/00001}{Data} \href{http://dx.doi.org/10.1126/science.1134239}{DOI} \vspace{2mm}

Xue G, Poldrack RA (2007). The neural substrates of visual perceptual learning of words: implications for the visual word form area hypothesis. \textit{J Cogn Neurosci, 19}, 1643-55. \href{http://dx.doi.org/10.1162/jocn.2007.19.10.1643}{DOI} \vspace{2mm}

\subsection*{2006}

Aron AR, Gluck MA, Poldrack RA (2006). Long-term test-retest reliability of functional MRI in a classification learning task. \textit{Neuroimage, 29}, 1000-6. \href{https://www.ncbi.nlm.nih.gov/pmc/articles/PMC1630684}{OA} \href{https://openneuro.org/datasets/ds000017/versions/00001}{Data} \href{http://dx.doi.org/10.1016/j.neuroimage.2005.08.010}{DOI} \vspace{2mm}

Aron AR, Poldrack RA (2006). Cortical and subcortical contributions to Stop signal response inhibition: role of the subthalamic nucleus. \textit{J. Neurosci., 26}, 2424-33. \href{https://www.ncbi.nlm.nih.gov/pmc/articles/PMC6793670}{OA} \href{http://dx.doi.org/10.1523/jneurosci.4682-05.2006}{DOI} \vspace{2mm}

Foerde K, Knowlton BJ, Poldrack RA (2006). Modulation of competing memory systems by distraction. \textit{Proc. Natl. Acad. Sci. U.S.A., 103}, 11778-83. \href{https://www.ncbi.nlm.nih.gov/pmc/articles/PMC1544246}{OA} \href{https://openneuro.org/datasets/ds000011/versions/00001}{Data} \href{http://dx.doi.org/10.1073/pnas.0602659103}{DOI} \vspace{2mm}

Poldrack RA (2006). Can cognitive processes be inferred from neuroimaging data? \textit{Trends Cogn. Sci. (Regul. Ed.), 10}, 59-63. \href{http://dx.doi.org/10.1016/j.tics.2005.12.004}{DOI} \vspace{2mm}

Poldrack RA, Willingham DB (2006). Functional neuroimaging of skill learning.. In \textit{Handbook of Neuroimaging of Cognition, 2nd Edition.} (p. 113-148). Cambridge, MA: MIT Press. \href{http://dx.doi.org/9780262033442}{DOI} \vspace{2mm}

Rodriguez PF, Aron AR, Poldrack RA (2006). Ventral-striatal/nucleus-accumbens sensitivity to prediction errors during classification learning. \textit{Hum Brain Mapp, 27}, 306-13. \href{http://dx.doi.org/10.1002/hbm.20186}{DOI} \vspace{2mm}

Seidman LJ, Thermenos HW, Poldrack RA, Peace NK, Koch JK, Faraone SV, Tsuang MT (2006). Altered brain activation in dorsolateral prefrontal cortex in adolescents and young adults at genetic risk for schizophrenia: an fMRI study of working memory. \textit{Schizophr. Res., 85}, 58-72. \href{http://dx.doi.org/10.1016/j.schres.2006.03.019}{DOI} \vspace{2mm}

Thermenos HW, Seidman LJ, Poldrack RA, Peace NK, Koch JK, Faraone SV, Tsuang MT (2006). RETRACTED: Elaborative Verbal Encoding and Altered Anterior Parahippocampal Activation in Adolescents and Young Adults at Genetic Risk for Schizophrenia Using fMRI. \textit{Biol. Psychiatry}. \href{http://dx.doi.org/10.1016/j.biopsych.2006.04.032}{DOI} \vspace{2mm}

\subsection*{2005}

Aron AR, Poldrack RA (2005). The cognitive neuroscience of response inhibition: relevance for genetic research in attention-deficit/hyperactivity disorder. \textit{Biol. Psychiatry, 57}, 1285-92. \href{http://dx.doi.org/10.1016/j.biopsych.2004.10.026}{DOI} \vspace{2mm}

Badre D, Poldrack RA, Paré-Blagoev EJ, Insler RZ, Wagner AD (2005). Dissociable controlled retrieval and generalized selection mechanisms in ventrolateral prefrontal cortex. \textit{Neuron, 47}, 907-18. \href{http://dx.doi.org/10.1016/j.neuron.2005.07.023}{DOI} \vspace{2mm}

Goldstein JM, Jerram M, Poldrack R, Ahern T, Kennedy DN, Seidman LJ, Makris N (2005). Hormonal cycle modulates arousal circuitry in women using functional magnetic resonance imaging. \textit{J. Neurosci., 25}, 9309-16. \href{https://www.ncbi.nlm.nih.gov/pmc/articles/PMC6725775}{OA} \href{http://dx.doi.org/10.1523/jneurosci.2239-05.2005}{DOI} \vspace{2mm}

Goldstein JM, Jerram M, Poldrack R, Anagnoson R, Breiter HC, Makris N, Goodman JM, Tsuang MT, Seidman LJ (2005). Sex differences in prefrontal cortical brain activity during fMRI of auditory verbal working memory. \textit{Neuropsychology, 19}, 509-19. \href{http://dx.doi.org/10.1037/0894-4105.19.4.509}{DOI} \vspace{2mm}

Katzir T, Misra M, Poldrack RA (2005). Imaging phonology without print: assessing the neural correlates of phonemic awareness using fMRI. \textit{Neuroimage, 27}, 106-15. \href{http://dx.doi.org/10.1016/j.neuroimage.2005.04.013}{DOI} \vspace{2mm}

Poldrack RA, Sabb FW, Foerde K, Tom SM, Asarnow RF, Bookheimer SY, Knowlton BJ (2005). The neural correlates of motor skill automaticity. \textit{J. Neurosci., 25}, 5356-64. \href{https://www.ncbi.nlm.nih.gov/pmc/articles/PMC6725010}{OA} \href{http://dx.doi.org/10.1523/jneurosci.3880-04.2005}{DOI} \vspace{2mm}

Stone WS, Thermenos HW, Tarbox SI, Poldrack RA, Seidman LJ (2005). Medial temporal and prefrontal lobe activation during verbal encoding following glucose ingestion in schizophrenia: A pilot fMRI study. \textit{Neurobiol Learn Mem, 83}, 54-64. \href{http://dx.doi.org/10.1016/j.nlm.2004.07.009}{DOI} \vspace{2mm}

Thermenos HW, Goldstein JM, Buka SL, Poldrack RA, Koch JK, Tsuang MT, Seidman LJ (2005). The effect of working memory performance on functional MRI in schizophrenia. \textit{Schizophr. Res., 74}, 179-94. \href{http://dx.doi.org/10.1016/j.schres.2004.07.021}{DOI} \vspace{2mm}

Trepel C, Fox CR, Poldrack RA (2005). Prospect theory on the brain? Toward a cognitive neuroscience of decision under risk. \textit{Brain Res Cogn Brain Res, 23}, 34-50. \href{http://dx.doi.org/10.1016/j.cogbrainres.2005.01.016}{DOI} \vspace{2mm}

Valera EM, Faraone SV, Biederman J, Poldrack RA, Seidman LJ (2005). Functional neuroanatomy of working memory in adults with attention-deficit/hyperactivity disorder. \textit{Biol. Psychiatry, 57}, 439-47. \href{http://dx.doi.org/10.1016/j.biopsych.2004.11.034}{DOI} \vspace{2mm}

\subsection*{2004}

Aron AR, Robbins TW, Poldrack RA (2004). Inhibition and the right inferior frontal cortex. \textit{Trends Cogn. Sci. (Regul. Ed.), 8}, 170-7. \href{http://dx.doi.org/10.1016/j.tics.2004.02.010}{DOI} \vspace{2mm}

Aron AR, Shohamy D, Clark J, Myers C, Gluck MA, Poldrack RA (2004). Human midbrain sensitivity to cognitive feedback and uncertainty during classification learning. \textit{J. Neurophysiol., 92}, 1144-52. \href{http://dx.doi.org/10.1152/jn.01209.2003}{DOI} \vspace{2mm}

Poldrack RA, Rodriguez P (2004). How do memory systems interact? Evidence from human classification learning. \textit{Neurobiol Learn Mem, 82}, 324-32. \href{http://dx.doi.org/10.1016/j.nlm.2004.05.003}{DOI} \vspace{2mm}

Poldrack RA, Sandak R (2004). Introduction to This Special Issue: The Cognitive Neuroscience of Reading \textit{Scientific Studies of Reading, 8}, 199-202. \href{http://dx.doi.org/10.1207/s1532799xssr0803\_1}{DOI} \vspace{2mm}

Shohamy D, Myers CE, Grossman S, Sage J, Gluck MA, Poldrack RA (2004). Cortico-striatal contributions to feedback-based learning: converging data from neuroimaging and neuropsychology. \textit{Brain, 127}, 851-9. \href{http://dx.doi.org/10.1093/brain/awh100}{DOI} \vspace{2mm}

Thermenos HW, Seidman LJ, Breiter H, Goldstein JM, Goodman JM, Poldrack R, Faraone SV, Tsuang MT (2004). Functional magnetic resonance imaging during auditory verbal working memory in nonpsychotic relatives of persons with schizophrenia: a pilot study. \textit{Biol. Psychiatry, 55}, 490-500. \href{http://dx.doi.org/10.1016/j.biopsych.2003.11.014}{DOI} \vspace{2mm}

\subsection*{2003}

Boas DA, Strangman G, Culver JP, Hoge RD, Jasdzewski G, Poldrack RA, Rosen BR, Mandeville JB (2003). Can the cerebral metabolic rate of oxygen be estimated with near-infrared spectroscopy? \textit{Phys Med Biol, 48}, 2405-18. \href{http://dx.doi.org/10.1088/0031-9155/48/15/311}{DOI} \vspace{2mm}

Jasdzewski G, Strangman G, Wagner J, Kwong KK, Poldrack RA, Boas DA (2003). Differences in the hemodynamic response to event-related motor and visual paradigms as measured by near-infrared spectroscopy. \textit{Neuroimage, 20}, 479-88. \href{http://dx.doi.org/10.1016/s1053-8119(03)00311-2}{DOI} \vspace{2mm}

Misra M, Katzir T, Wolf M, Poldrack RA (2003). Neural systems for rapid automatized naming identified using fMRI. \textit{Scientific Studies of Reading, 8.0}, 241-256. \href{http://dx.doi.org/xdhzgoip}{DOI} \vspace{2mm}

Poldrack RA, Packard MG (2003). Competition among multiple memory systems: converging evidence from animal and human brain studies. \textit{Neuropsychologia, 41}, 245-51. \href{http://dx.doi.org/10.1016/s0028-3932(02)00157-4}{DOI} \vspace{2mm}

Poldrack RA, Rodriguez P (2003). Sequence learning: what's the hippocampus to do? \textit{Neuron, 37}, 891-3. \href{http://dx.doi.org/10.1016/s0896-6273(03)00159-4}{DOI} \vspace{2mm}

Poldrack RA, Sandak R (2003). Introduction to special issue: The cognitive neuroscience of reading. \textit{Scientific Studies of Reading, 8.0}, 199-202. \href{http://dx.doi.org/nuwszbbp}{DOI} \vspace{2mm}

Sperling R, Chua E, Cocchiarella A, Rand-Giovannetti E, Poldrack R, Schacter DL, Albert M (2003). Putting names to faces: successful encoding of associative memories activates the anterior hippocampal formation. \textit{Neuroimage, 20}, 1400-10. \href{https://www.ncbi.nlm.nih.gov/pmc/articles/PMC3230827}{OA} \href{http://dx.doi.org/10.1016/s1053-8119(03)00391-4}{DOI} \vspace{2mm}

Temple E, Deutsch GK, Poldrack RA, Miller SL, Tallal P, Merzenich MM, Gabrieli JD (2003). Neural deficits in children with dyslexia ameliorated by behavioral remediation: evidence from functional MRI. \textit{Proc. Natl. Acad. Sci. U.S.A., 100}, 2860-5. \href{https://www.ncbi.nlm.nih.gov/pmc/articles/PMC151431}{OA} \href{http://dx.doi.org/10.1073/pnas.0030098100}{DOI} \vspace{2mm}

\subsection*{2002}

Golby AJ, Poldrack RA, Illes J, Chen D, Desmond JE, Gabrieli JD (2002). Memory lateralization in medial temporal lobe epilepsy assessed by functional MRI. \textit{Epilepsia, 43}, 855-63. \href{http://dx.doi.org/10.1046/j.1528-1157.2002.20501.x}{DOI} \vspace{2mm}

Poldrack RA (2002). Neural systems for perceptual skill learning. \textit{Behav Cogn Neurosci Rev, 1}, 76-83. \href{http://dx.doi.org/10.1177/1534582302001001005}{DOI} \vspace{2mm}

Poldrack RA, Paré-Blagoev EJ, Grant PE (2002). Pediatric functional magnetic resonance imaging: progress and challenges. \textit{Top Magn Reson Imaging, 13}, 61-70. \href{http://dx.doi.org/10.1097/00002142-200202000-00005}{DOI} \vspace{2mm}

\subsection*{2001}

Golby AJ, Poldrack RA, Brewer JB, Spencer D, Desmond JE, Aron AP, Gabrieli JD (2001). Material-specific lateralization in the medial temporal lobe and prefrontal cortex during memory encoding. \textit{Brain, 124}, 1841-54. \href{http://dx.doi.org/10.1093/brain/124.9.1841}{DOI} \vspace{2mm}

Poldrack RA, Clark J, Paré-Blagoev EJ, Shohamy D, Creso Moyano J, Myers C, Gluck MA (2001). Interactive memory systems in the human brain. \textit{Nature, 414}, 546-50. \href{https://openneuro.org/datasets/ds000052/versions/00001}{Data} \href{http://dx.doi.org/10.1038/35107080}{DOI} \vspace{2mm}

Poldrack RA, Gabrieli JD (2001). Characterizing the neural mechanisms of skill learning and repetition priming: evidence from mirror reading. \textit{Brain, 124}, 67-82. \href{http://dx.doi.org/10.1093/brain/124.1.67}{DOI} \vspace{2mm}

Poldrack RA, Temple E, Protopapas A, Nagarajan S, Tallal P, Merzenich M, Gabrieli JD (2001). Relations between the neural bases of dynamic auditory processing and phonological processing: evidence from fMRI. \textit{J Cogn Neurosci, 13}, 687-97. \href{http://dx.doi.org/10.1162/089892901750363235}{DOI} \vspace{2mm}

Temple E, Poldrack RA, Salidis J, Deutsch GK, Tallal P, Merzenich MM, Gabrieli JD (2001). Disrupted neural responses to phonological and orthographic processing in dyslexic children: an fMRI study. \textit{Neuroreport, 12}, 299-307. \href{http://dx.doi.org/10.1097/00001756-200102120-00024}{DOI} \vspace{2mm}

Vuilleumier P, Sagiv N, Hazeltine E, Poldrack RA, Swick D, Rafal RD, Gabrieli JD (2001). Neural fate of seen and unseen faces in visuospatial neglect: a combined event-related functional MRI and event-related potential study. \textit{Proc. Natl. Acad. Sci. U.S.A., 98}, 3495-500. \href{https://www.ncbi.nlm.nih.gov/pmc/articles/PMC30681}{OA} \href{http://dx.doi.org/10.1073/pnas.051436898}{DOI} \vspace{2mm}

Wagner AD, Paré-Blagoev EJ, Clark J, Poldrack RA (2001). Recovering meaning: left prefrontal cortex guides controlled semantic retrieval. \textit{Neuron, 31}, 329-38. \href{http://dx.doi.org/10.1016/s0896-6273(01)00359-2}{DOI} \vspace{2mm}

\subsection*{2000}

Hazeltine E, Poldrack R, Gabrieli JD (2000). Neural activation during response competition. \textit{J Cogn Neurosci, 12 Suppl 2}, 118-29. \href{http://dx.doi.org/10.1162/089892900563984}{DOI} \vspace{2mm}

Klingberg T, Hedehus M, Temple E, Salz T, Gabrieli JD, Moseley ME, Poldrack RA (2000). Microstructure of temporo-parietal white matter as a basis for reading ability: evidence from diffusion tensor magnetic resonance imaging. \textit{Neuron, 25}, 493-500. \href{http://dx.doi.org/10.1016/s0896-6273(00)80911-3}{DOI} \vspace{2mm}

Poldrack RA (2000). Imaging brain plasticity: conceptual and methodological issues--a theoretical review. \textit{Neuroimage, 12}, 1-13. \href{http://dx.doi.org/10.1006/nimg.2000.0596}{DOI} \vspace{2mm}

Seger CA, Poldrack RA, Prabhakaran V, Zhao M, Glover GH, Gabrieli JD (2000). Hemispheric asymmetries and individual differences in visual concept learning as measured by functional MRI. \textit{Neuropsychologia, 38}, 1316-24. \href{http://dx.doi.org/10.1016/s0028-3932(00)00014-2}{DOI} \vspace{2mm}

Temple E, Poldrack RA, Protopapas A, Nagarajan S, Salz T, Tallal P, Merzenich MM, Gabrieli JD (2000). Disruption of the neural response to rapid acoustic stimuli in dyslexia: evidence from functional MRI. \textit{Proc. Natl. Acad. Sci. U.S.A., 97}, 13907-12. \href{https://www.ncbi.nlm.nih.gov/pmc/articles/PMC17674}{OA} \href{http://dx.doi.org/10.1073/pnas.240461697}{DOI} \vspace{2mm}

\subsection*{1999}

Demb JB, Poldrack RA, Gabrieli JDE (1999). Functional neuroimaging of word processing in normal and dyslexic readers.. In \textit{Converging Methods for Understanding Reading and Dyslexia.} (p. 243-304). Cambridge, MA: MIT Press. \href{http://dx.doi.org/9780262519250}{DOI} \vspace{2mm}

Illes J, Francis WS, Desmond JE, Gabrieli JD, Glover GH, Poldrack R, Lee CJ, Wagner AD (1999). Convergent cortical representation of semantic processing in bilinguals. \textit{Brain Lang, 70}, 347-63. \href{http://dx.doi.org/10.1006/brln.1999.2186}{DOI} \vspace{2mm}

Poldrack RA, Prabhakaran V, Seger CA, Gabrieli JD (1999). Striatal activation during acquisition of a cognitive skill. \textit{Neuropsychology, 13}, 564-74. \href{http://dx.doi.org/10.1037//0894-4105.13.4.564}{DOI} \vspace{2mm}

Poldrack RA, Prabhakaran V, Seger CA, Gabrieli JDE (1999). Striatal activation during acquisition of a cognitive skill. \textit{Neuropsychology, 13}, 564-574. \href{http://dx.doi.org/10.1037/0894-4105.13.4.564}{DOI} \vspace{2mm}

Poldrack RA, Selco SL, Field JE, Cohen NJ (1999). The relationship between skill learning and repetition priming: experimental and computational analyses. \textit{J Exp Psychol Learn Mem Cogn, 25}, 208-35. \href{http://dx.doi.org/10.1037//0278-7393.25.1.208}{DOI} \vspace{2mm}

Poldrack RA, Wagner AD, Prull MW, Desmond JE, Glover GH, Gabrieli JD (1999). Functional specialization for semantic and phonological processing in the left inferior prefrontal cortex. \textit{Neuroimage, 10}, 15-35. \href{http://dx.doi.org/10.1006/nimg.1999.0441}{DOI} \vspace{2mm}

\subsection*{1998}

Gabrieli JD, Brewer JB, Poldrack RA (1998). Images of medial temporal lobe functions in human learning and memory. \textit{Neurobiol Learn Mem, 70}, 275-83. \href{http://dx.doi.org/10.1006/nlme.1998.3853}{DOI} \vspace{2mm}

Gabrieli JD, Poldrack RA, Desmond JE (1998). The role of left prefrontal cortex in language and memory. \textit{Proc. Natl. Acad. Sci. U.S.A., 95}, 906-13. \href{https://www.ncbi.nlm.nih.gov/pmc/articles/PMC33815}{OA} \href{http://dx.doi.org/10.1073/pnas.95.3.906}{DOI} \vspace{2mm}

Poldrack RA, Desmond JE, Glover GH, Gabrieli JD (1998). The neural basis of visual skill learning: an fMRI study of mirror reading. \textit{Cereb. Cortex, 8}, 1-10. \href{http://dx.doi.org/10.1093/cercor/8.1.1}{DOI} \vspace{2mm}

Poldrack RA, Gabrieli JD (1998). Memory and the brain: what's right and what's left? \textit{Cell, 93}, 1091-3. \href{http://dx.doi.org/10.1016/s0092-8674(00)81451-8}{DOI} \vspace{2mm}

Poldrack RA, Logan GD (1998). What is the mechanism for fluency in successive recognition? \textit{Acta Psychol (Amst), 98}, 167-81. \href{http://dx.doi.org/10.1016/s0001-6918(97)00041-3}{DOI} \vspace{2mm}

Wagner AD, Poldrack RA, Eldridge LL, Desmond JE, Glover GH, Gabrieli JD (1998). Material-specific lateralization of prefrontal activation during episodic encoding and retrieval. \textit{Neuroreport, 9}, 3711-7. \href{http://dx.doi.org/10.1097/00001756-199811160-00026}{DOI} \vspace{2mm}

\subsection*{1997}

Cohen NJ, Poldrack RA, Eichenbaum H (1997). Memory for items and memory for relations in the procedural/declarative memory framework. \textit{Memory, 5}, 131-78. \href{http://dx.doi.org/10.1080/741941149}{DOI} \vspace{2mm}

Poldrack RA, Gabrieli JD (1997). Functional anatomy of long-term memory. \textit{J Clin Neurophysiol, 14}, 294-310. \href{http://dx.doi.org/10.1097/00004691-199707000-00003}{DOI} \vspace{2mm}

Poldrack RA, Logan GD (1997). Fluency and response speed in recognition judgments. \textit{Mem Cognit, 25}, 1-10. \href{http://dx.doi.org/10.3758/bf03197280}{DOI} \vspace{2mm}

\subsection*{1996}

Poldrack RA (1996). On testing for stochastic dissociations. \textit{Psychon Bull Rev, 3}, 434-48. \href{http://dx.doi.org/10.3758/bf03214547}{DOI} \vspace{2mm}

